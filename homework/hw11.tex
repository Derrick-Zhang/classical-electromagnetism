\documentclass[12pt]{article}
\usepackage{fullpage}

\usepackage{amsmath,amssymb}
\usepackage{hyperref}
\usepackage{empheq}

\newcommand{\Z}{\mathbb{Z}}
\newcommand{\x}{{\bf x}}
\newcommand{\p}{{\bf p}}
\newcommand{\lb}{\left(}
\newcommand{\rb}{\right)}
\newcommand{\mb}{{m \pi/\beta}}
\newcommand{\B}{{\bf B}}
\newcommand{\bl}{{\bf l}}
\newcommand{\bJ}{{\bf J}}
\newcommand{\A}{{\bf A}}


\begin{document}
\begin{center}
{\bf Phys 5405}\\
HW 11 \\
5.13, 5.17, 5.19(a), 5.20, 5.21, 5.26, 5.27

\end{center}
\noindent{\bf 5.13} We can write down the current density,
\begin{equation}
    \bJ(\x) = \sigma \omega a \sin\theta \delta(r - a) \hat{\phi}~.
\end{equation}
Then the vector potential is given by
\begin{equation}
    \A(\x) = \frac{\mu_0}{4\pi} \int d^3 x'\, \frac{\bJ(\x')}{|\x - \x'|} = \frac{\mu_0 \sigma}{4\pi} \omega a \int r^2 dr \sin\theta' d \theta' d \phi' \frac{\sin \theta' \delta (r' - a) \hat{\phi'}}{|\x - \x'|}~.
\end{equation}
We can expand the Green function in spherical coordinates
\begin{equation}
    \frac{1}{|\x - \x'|} = \sum_{l=0}^\infty \sum_{m = -l}^l \frac{4\pi}{2l+1} \frac{r_<^l}{r_>^{l+1}} Y^*_{lm}(\theta', \phi') Y_{lm}(\theta, \phi)~,
\end{equation}
where $r_< = \min(r, r')$ and $r_> = \max(r, r')$.
Therefore, we get
\begin{align}
    \A(\x) = \mu_0 \sigma\omega a^3 \sum_{l,m} \frac{1}{2l+1} \frac{r_<^l}{r_>^{l+1}} Y_{lm}(\theta, \phi)\int d\Omega' Y_{lm}^*(\theta', \phi') \sin\theta' \hat{\phi}'~,
\end{align}
where $r_< = \min(r, a)$ and $r_> = \max(r, a)$.
Now we decompose $\hat{\phi}'$ in Cartesian coordinates,
\begin{equation}
    \hat{\phi}' = \cos\phi' \hat{i} + \sin\phi' \hat{j}~.
\end{equation}
We can evaluate the integrals,
\begin{align}
    \int d\Omega' Y^*_{lm}(\theta', \phi') \sin\theta' \cos \phi' &= \sqrt{\frac{2\pi}{3}}\int d\Omega' Y^*_{lm}(\theta', \phi') (-Y_{11}(\theta', \phi') + Y_{1,-1}(\theta', \phi'))\cr
    &=\sqrt{\frac{2\pi}{3}} (-\delta_{l1} \delta_{m1} + \delta_{l1}\delta_{m,-1})~.\\
    \int d\Omega' Y^*_{lm}(\theta', \phi') \sin\theta' \sin \phi' &= \sqrt{\frac{2\pi}{3}} i \int d\Omega' Y^*_{lm}(\theta', \phi') (Y_{11}(\theta', \phi') + Y_{1,-1}(\theta', \phi'))\cr
    &= \sqrt{\frac{2\pi}{3}} i (\delta_{l1}\delta_{m1} + \delta_{l1}\delta_{m,-1})~.
\end{align}
Therefore, we have
\begin{equation}
    \A(\x) = \frac{1}{3} \mu_0 \sigma \omega a^3 \frac{r_<}{r_>^2} \sin\theta (\cos \phi \hat{i} + \sin \phi \hat{j}) =  \frac{1}{3} \mu_0 \sigma \omega a^3 \frac{r_<}{r_>^2} \sin\theta \hat{\phi}~.
\end{equation}
Then, inside the sphere $r_< = r$ and $r_> = a$,
\begin{equation}
    \boxed{
    \A(\x) = \frac{1}{3} \mu_0 \sigma \omega a r \sin\theta \hat{\phi}}
\end{equation}
Outside the sphere, $r_< = a$ and $r_> = r$,
\begin{equation}
    \boxed{
    \A(\x) = \frac{1}{3} \mu_0 \sigma \omega \frac{a^4}{r^2} \sin\theta \hat{\phi}}
\end{equation}
The magnetic flux density is given by
\begin{equation}
    \B = \nabla \times \A = \frac{1}{r \sin \theta} \frac{\partial }{\partial \theta}(\sin \theta A_\phi) \hat{r} - \frac{1}{r} \frac{\partial}{\partial r}(r A_\phi) \hat{\theta}~.
\end{equation}
So inside the sphere, we have
\begin{equation}
    \boxed{
    \B = \frac{2}{3} \mu_0 \sigma \omega a (\cos \theta \hat{r} - \sin \theta \hat{\theta})}
\end{equation}
Outside the sphere, we have
\begin{equation}
    \boxed{
        \B = \frac{1}{3} \mu_0 \sigma \omega \frac{a^4}{r^3} (2\cos \theta \hat{r} + \sin \theta \hat{\theta})
    }
\end{equation}


\newpage
\noindent{\bf 5.17} For $z > 0$, the magnetic induction is generated by the current $\bJ$ and the image current $\bJ^*$,
\begin{equation}
    \B^+(\x) = \frac{\mu_0}{4\pi} \int d^3x' \frac{(\bJ(\x') + \bJ^*(\x')) \times (\x - \x')}{|\x - \x'|^3}~.
\end{equation}
It is an integration over the whole region.
For $z<0$, the magnetic induction is generated by the current $k \bJ$, where $k$ is a scaling constant because of different permeability,
\begin{equation}
    \B^-(\x) = \frac{\mu_0 \mu_r k}{4\pi} \int d^3 x' \frac{\bJ(\x') \times (\x - \x')}{|\x - \x'|^3}~.
\end{equation}
It is an integration over the region where $z' > 0$. Now we want to transform the first integral such that these two integrals have the same integration domain. Suppose the component of $\x'$ is $(x', y', z')$, then we define $\x'' = (x', y', -z')$ and we can write
\begin{equation}
    \B^+(\x) = \frac{\mu_0}{4\pi} \int d^3x' \frac{\bJ(\x') \times (\x - \x')}{|\x - \x'|^3} + \frac{\mu_0}{4\pi} \int d^3 x' \frac{\bJ^*(\x'') \times (\x - \x'')}{|\x - \x''|^3}~.
\end{equation}
Now this integral is defined in the region where $z' > 0$. And especially, when $z = 0$, there is no difference between $|\x - \x'|$ and $|\x - \x''|$.


Now the boundary conditions are given by
\begin{equation}
    \B^+_z(z = 0) = \B^-_z(z = 0)~, \quad \B^+_{x,y}(z = 0) = \frac{1}{\mu_r} \B^-_{x,y}(z = 0)~.
\end{equation}
For the first equation, we can equate the numerator in the integrand. When $z = 0$,
\begin{equation}
    \hat{z} \cdot [\bJ(\x')  \times (\x - \x')] + \hat{z} \cdot [\bJ^*(\x'')  \times (\x - \x'')] = \mu_r k \hat{z} \cdot (\bJ(\x') \times (\x - \x'))~,
\end{equation}
from which we can get that at $z = 0$,
\begin{align}
    (\x - \x')\cdot [\hat{z} \times \bJ(\x')] + (\x - \x'') \cdot [\hat{z} \times \bJ^*(\x'')] = \mu_r k (\x - \x') \cdot (\hat{z}  \times \bJ(\x'))~,
\end{align}
and expanding it in components we can get
\begin{equation}\label{eqn:eqns1}
    \boxed{
     J_y(\x') + J^*_y(\x'') = \mu_r k J_y(\x'), \quad J_x(\x') + J^*_x(\x'') = \mu_r k J_x(\x')}
\end{equation}
Another equation gives that at $z = 0$,
\begin{align}
    \hat{z} \times [\bJ(\x') \times (\x - \x')] + \hat{z} \times [\bJ^*(\x'') \times (\x - \x'')] = k \hat{z} \times [\bJ(\x') \times (\x - \x')]~.
\end{align}
Now using,
\begin{equation}
    \A \times (\B \times {\bf C}) = (\A \cdot {\bf C}) \B - (\A \cdot \B) {\bf C}~,
\end{equation}
we can further simplify the above equation to,
\begin{equation}
    J_z^*(\x'')(\x - \x'') - z' \bJ^*(\x'') =(k-1) J_z(\x') (\x - \x') + (k-1) z'  \bJ(\x')~.
\end{equation}
Expanding it into components and using the fact that the equation holds for arbitrary $\x$, we have,
\begin{equation}\label{eqn:eqns2}
    \boxed{
    J_x^*(\x'') = (1-k) J_x(\x'), \quad J_y^*(\x'') = (1-k) J_y(\x'), \quad J_z^*(\x'') = (k-1) J_z(\x')}
\end{equation}
From \eqref{eqn:eqns1} and \eqref{eqn:eqns2}, we can solve for $k = 2/(1 + \mu_r)$. Plug it back into \eqref{eqn:eqns2}, we have the image current distribution $\bJ^*$, with components,
\begin{equation}\nonumber
    \boxed{
        \left(\frac{\mu_r - 1}{\mu_r + 1}\right) J_x(x,y,-z), \quad  \left(\frac{\mu_r - 1}{\mu_r + 1}\right) J_y(x,y,-z), \quad  -\left(\frac{\mu_r - 1}{\mu_r + 1}\right) J_z(x,y,-z)
    }
\end{equation}
Since $k = 2/(1 + \mu_r)$, we have stated that for $z < 0$, the magnetic induction is due to a current distribution $k \bJ$ in a medium of relative permeability $\mu_r$. We can also consider it due to a current distribution
\begin{equation}
    \boxed{
    k\mu_r \bJ = \frac{2\mu_r}{1 + \mu_r} \bJ
    }
\end{equation}
in a medium of unit relative permeability.

\newpage
\noindent{\bf 5.19 (a)} Since $\bJ = 0$, we can use the magnetic scalar potential $\Phi_M$. Since the magnetization is uniform, we have
\begin{equation}
    \Phi_M(\x) = \frac{1}{4\pi} \oint_S \frac{{\bf n}' \cdot \hat{z}\, M_0 \, da' }{|\x - \x'|}
\end{equation}
Since the magnetization points along the $z$-direction, we only need to consider the top boundary (say, at $z = L$) and the bottom boundary (at $z = 0$) and let the axis of the cylinder lying in the $z$-axis. Then when $\x$ is on the axis, at top, $|\x - \x'| = \sqrt{x'^2 + y'^2 + (z - L)^2}$ and at bottom, $|\x - \x'| = \sqrt{x'^2 + y'^2 + z^2}$. For the two dimensional surface integral, we can also use polar coordinates. Then, we have,
\begin{align}
    \Phi_M(\x) &= \frac{M_0}{4\pi} \int_0^a \rho' d\rho' \int_0^{2\pi} d\phi \left(\frac{1}{\sqrt{\rho'^2 + (z - L)^2}} - \frac{1}{\sqrt{\rho'^2 + z^2}}\right)\cr
    &= \frac{M_0}{2} \int_0^a d\rho' \left(\frac{\rho'}{\sqrt{\rho'^2 + (z - L)^2}} - \frac{\rho'}{\sqrt{\rho'^2 + z^2}}\right)\cr
    &= \frac{M_0}{2} \left(\sqrt{\rho'^2 + (z-L)^2} - \sqrt{\rho'^2 + z^2} \right)\Big|_{\rho' = 0}^{\rho' = a}\cr
    &= \frac{M_0}{2} \left(\sqrt{a^2 + (z-L)^2} - |z - L| - \sqrt{a^2 + z^2} + |z|\right)~.
\end{align}
Therefore,
\begin{equation}
    \Phi_M(z) = \begin{cases}
        \frac{M_0}{2}\left(\sqrt{a^2 + (z-L)^2} - \sqrt{a^2 + z^2} - L\right) & z < 0\\
        \frac{M_0}{2}\left(\sqrt{a^2 + (z-L)^2} - \sqrt{a^2 + z^2} + 2z- L\right) & 0 < z < L\\
        \frac{M_0}{2}\left(\sqrt{a^2 + (z-L)^2} - \sqrt{a^2 + z^2} + L\right) & z > L
    \end{cases}
\end{equation}
The magnetic field ${\bf H}$ is given by ${\bf H} = - \nabla \Phi_M = - \hat{z}\,\partial \Phi_M/ \partial z$. Then
\begin{align}
    {\bf H}_{\rm in} &= -\frac{M_0}{2} \left(\frac{z-L}{\sqrt{a^2 + (z-L)^2} } - \frac{z}{\sqrt{a^2 + z^2}} + 2\right)\hat{z}\\
    {\bf H}_{\rm out} &= -\frac{M_0}{2} \left(\frac{z-L}{\sqrt{a^2 + (z-L)^2} } - \frac{z}{\sqrt{a^2 + z^2} }\right) \hat{z}
\end{align}
The magnetic induction is given by $\B = \mu_0 ({\bf H} + {\bf M})$. So we have,
\begin{align}
    \B_{\rm in} = -\frac{\mu_0 M_0}{2} \left(\frac{z-L}{\sqrt{a^2 + (z-L)^2} } - \frac{z}{\sqrt{a^2 + z^2}} \right) \hat{z} \\
    \B_{\rm out} = -\frac{\mu_0 M_0}{2} \left(\frac{z-L}{\sqrt{a^2 + (z-L)^2} } - \frac{z}{\sqrt{a^2 + z^2}} \right) \hat{z}
\end{align}

\newpage
\noindent{\bf 5.20} We start from the force equation
\begin{equation}
    {\bf F} = \int \bJ(\x) \times \B(\x)\, d^3x~.
\end{equation}
Using the fact that a magnetization $\bf M$ inside a volume $V$ bounded by a surface $S$ is equivalent to a volume current density $\bJ_M = \nabla \times {\bf M}$ and a surface current density ${\bf M} \times {\bf n}$, we can write
\begin{align}
    {\bf F} = \int_V (\nabla \times {\bf M})\times \B_e \,d^3 x + \int_S ({\bf M} \times {\bf n}) \times \B_e \,d a
\end{align}
Since
\begin{equation}
    \nabla(\A \cdot \B) = (\A \cdot \nabla) \B + (\B \cdot \nabla) \A + \A \times (\nabla \times \B) + \B \times (\nabla \times \A)~.
\end{equation}
In our case, we have
\begin{equation}
    (\nabla \times {\bf M}) \times \B_e = - \nabla({\bf M} \cdot \B) + ({\bf M} \cdot \nabla) \B_e + (\B_e \cdot \nabla) {\bf M} + {\bf M} \times (\nabla \times \B_e)~.
\end{equation}
Since there's no external current, $\nabla \times \B_e = 0$, we have
\begin{equation}\label{eqn:using-Stokes}
    \boxed{(\nabla \times {\bf M}) \times \B_e = - \nabla({\bf M} \cdot \B_e) + ({\bf M} \cdot \nabla) \B_e + (\B_e \cdot \nabla) {\bf M}}
\end{equation}
Also, we have
\begin{equation}\label{eqn:using-Stokes2}
    \boxed{
    ({\bf M} \times {\bf n}) \times \B_e = (\B_e \cdot {\bf M}) {\bf n} - (\B_e \cdot {\bf n}) {\bf M}}
\end{equation}
For volume integration of the first term in the right hand side of \eqref{eqn:using-Stokes}, we can use Stokes theorem, which will cancel the surface integral of the first term in the right hand side of \eqref{eqn:using-Stokes2}. So we are left with
\begin{equation}
    {\bf F} = \int_V ({\bf M} \cdot \nabla) \B_e\, d^3 x+ \int_V (\B_e \cdot \nabla){\bf M} \, d^3 x - \int_S (\B_e \cdot {\bf n}) {\bf M} \, da~.
\end{equation}
Now we want to do integration by parts, we have
\begin{equation}
    \int_V(\A \cdot \nabla) \B\, d^3x = - \int_V(\nabla \cdot \A) \B\, d^3x + \int_S ( \A\cdot{\bf n} )\B \, da~.
\end{equation}
Then we have
\begin{equation}
    \boxed{
    {\bf F} = - \int_V (\nabla \cdot {\bf M}) \B_e \,d^3x + \int_S({\bf M} \cdot{\bf n} ) \B_e \, da
    }~,
\end{equation}
where we have used the fact that $\nabla \cdot \B_e = 0$.

The force is then

\newpage
\noindent{\bf 5.21}

\newpage
\noindent{\bf 5.26}

\newpage
\noindent{\bf 5.27}



\end{document}