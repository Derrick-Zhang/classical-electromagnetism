\documentclass[12pt]{article}
\usepackage{fullpage}

\usepackage{amsmath,amssymb}
\usepackage{hyperref}

\newcommand{\Z}{\mathbb{Z}}

\begin{document}

\begin{center}
{\bf Phys 5405}\\
HW 6 \\
3.13 3.14 3.16 b,c,d
\end{center}
{\bf 3.13}
The Green function for a spherical shell bounded by $r = a$ and $r = b$ is
\begin{equation}
    G(\vec x, \vec x') = 4\pi \sum_{l = 0}^\infty \sum_{m = -l}^l \frac{Y^*_{lm}(\theta', \phi') Y_{lm}(\theta, \phi)}{(2l+1)\left[1- \left(\frac{a}{b}\right)^{2l+1}\right]}\left(r^l_< - \frac{a^{2l+1}}{r^{l+1}_<}\right)\left(\frac{1}{r^{l+1}_>} - \frac{r^l_>}{b^{2l+1}}\right)~.
\end{equation}
Since we have azimuthal symmetry, we only need to consider $m = 0$, therefore,
\begin{equation}
    G(\vec x, \vec x') = \sum_{l = 0}^\infty\frac{P_l(\cos \theta') P_l(\cos \theta)}{1- \left(\frac{a}{b}\right)^{2l+1}}\left(r^l_< - \frac{a^{2l+1}}{r^{l+1}_<}\right)\left(\frac{1}{r^{l+1}_>} - \frac{r^l_>}{b^{2l+1}}\right)~.
\end{equation}
Now the potential is
\begin{equation}
    \Phi(\vec x) = \frac{1}{4\pi \epsilon_0} \int_V \rho(\vec x') G(\vec x, \vec x') d^3 x'+ \frac{1}{4\pi} \int_S\left[G(\vec x, \vec x') \frac{\partial \Phi}{\partial n'} - \Phi(\vec x') \frac{\partial G(\vec x, \vec x')}{\partial n'}\right]da'~.
\end{equation}
In between the two spheres, the charge density is zero, and the Green function vanishes on the boundary. We have
\begin{equation}
    \Phi(\vec x) = - \frac{1}{4\pi} \int_S \Phi(\vec x') \frac{\partial G(\vec x, \vec x')}{\partial n'} da'~.
\end{equation}
Then we have to calculate the derivative of the Green function. For the boundary at radius $a$, we have $r > r'$,
\begin{equation}
    G(\vec x, \vec x') = \sum_{l=0}^\infty  \frac{P_l(\cos \theta') P_l(\cos \theta)}{1 - \left(\frac{a}{b}\right)^{2l+1}} \left(r'^l - \frac{a^{2l+1}}{r'^{l+1}}\right) \left(\frac{1}{r^{l+1}} - \frac{r^l}{b^{2l+1}}\right)~.
\end{equation}
Calculate its derivative and evaluate it at $r' = a$
\begin{align}
    \frac{\partial G(\vec x, \vec x')}{\partial n'}\Bigg|_{r' = a} &= - \frac{\partial G(\vec x, \vec x')}{\partial r'} \Bigg |_{r' = a} \cr
    &= -\sum_{l=0}^\infty  \frac{P_l (\cos \theta')P_l(\cos \theta)}{1-\left(\frac{a}{b}\right)^{2l+1}}\left(l r'^{l-1} +(l+1)\frac{a^{2l+1}}{r'^{l+2}}\right) \left(\frac{1}{r^{l+1}} - \frac{r^l}{b^{2l+1}}\right) \Bigg |_{r' = a} \cr
    &=-\sum_{l=0}^\infty  \frac{P_l (\cos \theta')P_l(\cos \theta)}{1-\left(\frac{a}{b}\right)^{2l+1}}a^{l-1}(2l+1) \left(\frac{1}{r^{l+1}} - \frac{r^l}{b^{2l+1}}\right)~.
\end{align}
Then
\begin{align}
    \int_{S_a}&\Phi(\vec x') \frac{\partial G(\vec x, \vec x')}{\partial n'}da' = \int_0^{\pi/2} a^2 \sin\theta' d\theta' \int_0^{2\pi} d\phi'\, V \frac{\partial G(\vec x, \vec x')}{\partial n'} \cr
    &= - 2 \pi V \sum_{l=0}^\infty \frac{P_l(\cos \theta)}{1 - (a/b)^{2l+1}} a^{l+1} (2l+1) \left(\frac{1}{r^{l+1}} - \frac{r^l}{b^{2l+1}}\right) \int_0^{\pi/2} P_l(\cos\theta') \sin\theta' d \theta'
\end{align}
Now evaluate for $l \neq 0$,
\begin{equation}
    \int_0^{\pi/2} P_l(\cos \theta') \sin \theta' d \theta' = \int_0^1 P_l(x) dx = \frac{1}{2l+1}\left[P_{l-1}(0) - P_{l+1}(0)\right]
\end{equation}
For $l = 0$, we simply have
\begin{equation}
    \int_0^{\pi/2} P_l(\cos \theta') \sin \theta' d \theta' = \int_0^1 P_l(x) dx = 1~.
\end{equation}
Therefore,
\begin{align}
    &\int_{S_a}\Phi(\vec x') \frac{\partial G(\vec x, \vec x')}{\partial n'}da' \cr
    &= - 2 \pi V \frac{a}{1 - (a/b)} \left(\frac{1}{r} - \frac{1}{b}\right) - 2 \pi V \sum_{l=1}^\infty \frac{P_l(\cos \theta) a^{l+1}}{1 - (a/b)^{2l+1}}   \left(\frac{1}{r^{l+1}} - \frac{r^l}{b^{2l+1}}\right) \left[P_{l-1}(0) - P_{l+1}(0)\right]~.\nonumber
\end{align}
Similarly, for boundary at radius $b$, we have $r < r'$,
\begin{equation}
    G(\vec x, \vec x') = \sum_{l=0}^\infty  \frac{P_l(\cos \theta') P_l(\cos \theta)}{1 - \left(\frac{a}{b}\right)^{2l+1}} \left(r^l - \frac{a^{2l+1}}{r^{l+1}}\right) \left(\frac{1}{r'^{l+1}} - \frac{r'^l}{b^{2l+1}}\right)~.
\end{equation}
Calculate its derivative and evaluate it at $r' = b$
\begin{align}
    \frac{\partial G(\vec x, \vec x')}{\partial n'}\Bigg|_{r' = b} &= \frac{\partial G(\vec x, \vec x')}{\partial r'} \Bigg |_{r' = b} \cr
    &= \sum_{l=0}^\infty  \frac{P_l (\cos \theta')P_l(\cos \theta)}{1-\left(\frac{a}{b}\right)^{2l+1}}\left(r^l - \frac{a^{2l+1}}{r^{l+1}}\right) \left(-(l+1)\frac{1}{r'^{l+2}} - l\frac{r'^{l-1} }{b^{2l+1}}\right) \Bigg |_{r' = b} \cr
    &=- \sum_{l=0}^\infty  \frac{P_l (\cos \theta')P_l(\cos \theta)}{1-\left(\frac{a}{b}\right)^{2l+1}}\left(r^l - \frac{a^{2l+1}}{r^{l+1}}\right) (2l+1) b^{-l-2}~.
\end{align}
Then
\begin{align}
    \int_{S_b}&\Phi(\vec x') \frac{\partial G(\vec x, \vec x')}{\partial n'}da' = \int_{\pi/2}^\pi b^2 \sin\theta' d\theta' \int_0^{2\pi} d\phi'\, V \frac{\partial G(\vec x, \vec x')}{\partial n'} \cr
    &= - 2 \pi V \sum_{l=0}^\infty \frac{P_l(\cos \theta)}{1 - (a/b)^{2l+1}} b^{-l} (2l+1) \left(r^l - \frac{a^{2l+1}}{r^{l+1}}\right) \int_{\pi/2}^\pi P_l(\cos\theta') \sin\theta' d \theta'\cr
    &= -2\pi V \frac{1}{1-a/b}(1-a/r) - 2\pi V \sum_{l=1}^\infty (-1)^l \frac{P_l(\cos \theta) b^{-l}}{1-(a/b)^{2l+1}} \left(r^l - \frac{a^{2l+1}}{r^{l+1}}\right)[P_{l-1}(0) - P_{l+1}(0)]\nonumber
\end{align}
Then the potential between the two spheres are simply given by
\begin{align}
    \Phi(\vec x) =&~ - \frac{1}{4\pi} \int_{S_a} \Phi(\vec x') \frac{\partial G(\vec x, \vec x')}{\partial n'} da' - \frac{1}{4\pi} \int_{S_b} \Phi(\vec x') \frac{\partial G(\vec x, \vec x')}{\partial n'}da'\cr
    =&~ \frac{V}{2} \frac{1}{1-a/b}\left(\frac{a}{r} - \frac{a}{b}\right) + \frac{V}{2} \frac{1}{1-a/b}\left(1 - \frac{a}{r}\right) \cr
    &+ \frac{V}{2} \sum_{l = 1}^\infty [P_{l-1}(0)-P_{l+1}(0)]P_l(\cos \theta) \cr
    & \times \left[\frac{a^{l+1}}{1-(a/b)^{2l+1}}\left(\frac{1}{r^{l+1}} - \frac{r^l}{b^{2l+1}}\right) + (-1)^l \frac{b^{-l}}{1-(a/b)^{2l+1}} \left(r^l - \frac{a^{2l+1}}{r^{l+1}}\right)\right]\cr
    =&~ \frac{V}{2} + \frac{V}{2}\sum_{l=1}^\infty(P_{l-1}(0) - P_{l+1}(0)) P_l(\cos\theta) \left(
        \frac{(-1)^l b^{l+1} - a^{l+1} }{b^{2l+1} - a^{2l+1} }r^l + \frac{(-1)^l b^{-l}-a^{-l} }{b^{-(2l+1)} - a^{-(2l+1)} } r^{-(l+1)}
    \right)~.\nonumber
\end{align}
This is exactly what I got in homework 5, Jackson problem 3.1.

\newpage
\noindent{\bf 3.14 (a)} A line charge of length $2d$ with a total charge $Q$ has linear charge density varying as $(d^2 - z^2)$, where $z$ is the distance from the midpoint. A grounded, conducting, spherical shell of inner radius $b > d$ is centered at the midpoint of the line charge.

First, we write down the charge density as a function of $z$,
\begin{equation}
    \rho(z) = k(d^2 - z^2),
\end{equation}
where the constant $k$ has to be determined. We have
\begin{equation}
    Q= 2\int_0^d \rho(z) dz = 2\int_0^d k(d^2 - z^2)dz = \frac{4}{3}k d^3~.
\end{equation}
Therefore, $k = 3 Q /(4 d^3)$ and
\begin{equation}
    \rho(z) = \frac{3Q}{4d^3}(d^2 - z^2)~.
\end{equation}
Now we write the charge density as a function of the position vector, in terms of delta functions, for $|\vec x'| \le d$,
\begin{equation}
    \rho(\vec x') = \frac{3Q}{4d^3}(d^2 - r'^2) \frac{1}{2\pi r'^2} [\delta(\cos \theta' - 1) + \delta(\cos \theta' + 1)]~.
\end{equation}
Otherwise, the charge density is zero.
The Green function inside a sphere of radius $b$ with azimuthal symmetry is given by,
\begin{equation}
    G(\vec x, \vec x') = \sum_{l=0}^\infty P_l(\cos \theta') P_l(\cos \theta) r_<^l \left(\frac{1}{r_>^{l+1}} - \frac{r_>^l}{b^{2l+1}}\right)~.
\end{equation}
%Then its derivative evaluated at radius $b$
%\begin{equation}
%\frac{\partial G(\vec x, \vec x')}{\partial n'} \Bigg|_{r' = b}= \frac{\partial G(\vec x, \vec x')}{\partial r'} \Bigg|_{r' = b}= - \sum_{l = 0}^\infty P_l(\cos \theta') P_l(\cos \theta) \frac{r^l}{b^{l+2}} (2l+1)~.
%\end{equation}
Since on the boundary, the potential vanishes, we have
\begin{equation}
    \Phi(\vec x) = \frac{1}{4\pi \epsilon_0} \int_V \rho(\vec x') G(\vec x, \vec x') d^3 x'~.
\end{equation}

For $r > d$, we have $r_< = r'$ and $r_> = r$.
Then the potential is
\begin{align}
    \Phi(\vec x) =&~ \frac{1}{4\pi \epsilon_0}\frac{3 Q}{4d^3}\sum_{l =0}^\infty P_l(\cos\theta) \left(\frac{1}{r^{l+1}} - \frac{r^l}{b^{2l+1}}\right) \cr
    & \int_0^{d}dr' \int_{-1}^1 d(\cos \theta') (d^2 - r'^2)r'^l P_l(\cos \theta')
    [\delta(\cos \theta' -1) + \delta (\cos \theta' + 1)] \cr
    =&~ \frac{2}{4\pi \epsilon_0}\frac{3Q}{4d^3}\sum_{l = 0,~{\rm even}}^\infty P_l(\cos \theta) \left(\frac{1}{r^{l+1}} - \frac{r^l}{b^{2l+1}}\right)  \int_0^d dr'(d^2 - r'^2) r'^l \cr
    =&~ \frac{2}{4\pi \epsilon_0}\frac{3Q}{4d^3}\sum_{l = 0,~{\rm even}}^\infty P_l(\cos \theta) \left(\frac{1}{r^{l+1}} - \frac{r^l}{b^{2l+1}}\right)  \frac{2 d^{l+3}}{(l+1)(l+3)}\cr
    =&~ \frac{3Q}{4\pi \epsilon_0}\sum_{l = 0,~{\rm even}}^\infty P_l(\cos \theta) \left(\frac{1}{r^{l+1}} - \frac{r^l}{b^{2l+1}}\right)  \frac{d^{l}}{(l+1)(l+3)}~.
\end{align}

For $r < d$, we have two regimes, $r' < r$ and $r < r' < d$. Then in the first regime, its contribution to the potential is
\begin{align}
    \Phi_1(\vec x) &= \frac{2}{4\pi \epsilon_0} \frac{3Q}{4d^3} \sum_{l = 0,~{\rm even}}^\infty P_l(\cos \theta) \left(\frac{1}{r^{l+1}} - \frac{r^l}{b^{2l+1}}\right) \int_0^r dr' (d^2 - r'^2) r'^l\cr
    &=\frac{2}{4\pi \epsilon_0} \frac{3Q}{4d^3} \sum_{l = 0,~{\rm even}}^\infty P_l(\cos \theta) \left(\frac{1}{r^{l+1}} - \frac{r^l}{b^{2l+1}}\right) \left(\frac{d^2 r^{l+1} }{l+1} - \frac{r^{l+3}}{l+3} \right)~.
\end{align}
In the second regime, its contribution to the potential is
\begin{align}
    \Phi_2(\vec x) &= \frac{2}{4\pi \epsilon_0} \frac{3Q}{4d^3} \sum_{l = 0,~{\rm even}}^\infty P_l(\cos \theta) r^l \int_r^d dr' (d^2 - r'^2) \left(\frac{1}{r'^{l+1}} - \frac{r'^l}{b^{2l+1}}\right)~.
\end{align}
Denote the integral by
\begin{equation}
    \mathcal I_l = \int_r^d dr' (d^2 - r'^2) \left(\frac{1}{r'^{l+1}} - \frac{r'^l}{b^{2l+1}}\right)~.
\end{equation}
When $l = 0$, it's
\begin{equation}
    \mathcal I_0 = d^2 \ln \frac dr - \frac{d^2}{b}(d-r) - \frac 12(d^2 - r^2) + \frac{1}{3b}(d^3 - r^3)~.
\end{equation}
When $l = 2$, it's
\begin{equation}
    \mathcal I_2 = -\ln\frac{d}{r} - \frac{d^2}{2}(d^{-2} - r^{-2}) - \frac{d^2}{3b^5}(d^3 - r^3) + \frac{1}{5 b^5}(d^5 - r^5)~.
\end{equation}
When $l \ge 4$, it'
\begin{equation}
    \mathcal I_l = -\frac{d^2}{l}(d^{-l} - r^{-l}) - \frac{d^2}{(l+1) b^{2l+1}} (d^{l+1} - r^{l+1}) + \frac{1}{l-2} (d^{2-l} - r^{2-l}) + \frac{1}{(l+3) b^{2l+1}} (d^{l+3} - r^{l+3})~.
\end{equation}
Therefore, for $r< d$
\begin{equation}
    \Phi(\vec x) = \Phi_1(\vec x) + \Phi_2(\vec x)~.
\end{equation}

\newpage
\noindent{\bf 3.14 (b)} For potential near the boundary, we have $r> d$ and
\begin{equation}
    \Phi(\vec x) = \frac{3Q}{4\pi \epsilon_0} \sum_{l= 0,~{\rm even}}^\infty P_l(\cos \theta) \left(\frac{1}{r^{l+1}} - \frac{r^l}{b^{2l+1}}\right) \frac{d^l}{(l+1)(l+3)}~.
\end{equation}
Then, the surface charge density is given by
\begin{align}
    \sigma = \epsilon_0 \frac{\partial \Phi}{\partial r}\Bigg|_{r = b} = - \frac{3Q}{4\pi} \sum_{l = 0, ~{\rm even}}^\infty P_l(\cos\theta) \frac{(2l+1)}{(l+1)(l+3)} \frac{d^l}{b^{l+2}}~.
\end{align}

\newpage
\noindent{\bf 3.14 (c)} In the limit that $d \ll b$, then also $d << r$. The potential in (a) is
\begin{equation}
    \Phi(\vec x) = \frac{3Q}{4\pi \epsilon_0} \sum_{l= 0,~{\rm even}}^\infty P_l(\cos \theta) \left(\frac{1}{r^{l+1}} - \frac{r^l}{b^{2l+1}}\right) \frac{d^l}{(l+1)(l+3)}~.
\end{equation}
Taking the limit, only the $l = 0$ term matters,
\begin{equation}
    \Phi(\vec x) = \frac{Q}{4\pi \epsilon_0} \left(\frac{1}{r} - \frac{1}{b} \right)~.
\end{equation}
It consists the point charge potential and the induced surface charge potential.
Similarly, the surface charge density is
\begin{equation}
    \sigma = -\frac{Q}{4\pi b^2},
\end{equation}
from which we can see that the charge is uniformly distributed on the sphere with a total charge $Q$.


\end{document}