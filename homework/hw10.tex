\documentclass[12pt]{article}
\usepackage{fullpage}

\usepackage{amsmath,amssymb}
\usepackage{hyperref}
\usepackage{empheq}

\newcommand{\Z}{\mathbb{Z}}
\newcommand{\x}{{\bf x}}
\newcommand{\p}{{\bf p}}
\newcommand{\lb}{\left(}
\newcommand{\rb}{\right)}
\newcommand{\mb}{{m \pi/\beta}}
\newcommand{\B}{{\bf B}}
\newcommand{\bl}{{\bf l}}
\newcommand{\A}{{\bf A}}

\begin{document}
\begin{center}
{\bf Phys 5405}\\
HW 10 \\
5.1, 5.3, 5.4(a), 5.7(a,b,c,d), 5.8, 5.10(a,b)

\end{center}
\noindent{\bf 5.1}
We have the differential expression
\begin{equation}
    d {\bf B} = \frac{\mu_0 I}{4\pi}\, d\bl' \times \frac{\x - \x'}{|\x - \x'|^3}~.
\end{equation}
Then
\begin{equation}
    \B = \int d \B = \frac{\mu_0 I}{4\pi} \oint d \bl' \times \frac{\x - \x'}{|\x - \x'|^3}~.
\end{equation}
It suffices to show that
\begin{equation}
    \oint d\bl' \times \frac{\x - \x'}{|\x - \x'|^3} = \nabla \Omega~.
\end{equation}

First, we can make use of the identity,
\begin{equation}
    \boxed{
    \frac{\x - \x'}{|\x - \x'|^3} = \nabla' \left(\frac{1}{|\x - \x'|}\right) \equiv {\bf V}~.
    }
\end{equation}
Then, consider each components
\begin{align}
    \boxed{
    \hat{\x}_i \cdot \oint d\bl' \times {\bf V} = \oint d\bl' \cdot ({\bf V} \times \hat{\x}_i) = \int_S d{\bf a}' \cdot (\nabla' \times({\bf V} \times \hat{\x}_i))~.
    }
\end{align}
Now we evaluate $\nabla' \times({\bf V} \times \hat{\x}_i)$, using the identity
\begin{equation}
    \nabla \times ({\bf A} \times {\bf B}) = {\bf A}(\nabla \cdot {\bf B}) - {\bf B}(\nabla \cdot {\bf A}) + ({\bf B} \cdot \nabla) {\bf A} - ({\bf A} \cdot \nabla) {\bf B}~,
\end{equation}
so we have
\begin{equation}
    \boxed{
    \nabla' \times({\bf V} \times \hat{\x}_i) = - \hat{\x}_i(\nabla' \cdot {\bf V}) + (\hat{\x}_i \cdot \nabla') {\bf V} = (\hat{\x}_i \cdot \nabla') {\bf V} = \frac{\partial}{\partial x'_i} {\bf V}~,
    }
\end{equation}
where we have used the fact that for $\x \neq \x'$,
\begin{equation}
    \nabla' \cdot {\bf V} = \nabla'^2 \left(\frac{1}{|\x - \x'|} \right) = 0~.
\end{equation}
Finally, we have achieved that
\begin{equation}
    \hat{\x}_i \cdot \oint d\bl' \times \frac{\x - \x'}{|\x - \x'|^3} = \int_S d{\bf a}' \cdot \frac{\partial}{\partial x_i'} \left(\nabla' \left(\frac{1}{|\x - \x'|}\right)\right) = -\frac{\partial}{\partial x_i}\int_S d{\bf a}' \cdot \nabla' \left(\frac{1}{|\x - \x'|}\right)
\end{equation}
From Jackson, the equation below (1.25), we have
\begin{equation}
    d{\bf a}' \cdot \nabla' \left(\frac{1}{|\x - \x'|}\right) = - d\Omega~.
\end{equation}
Therefore,
\begin{equation}
    \boxed{
    \hat{\x}_i \cdot \oint d\bl' \times \frac{\x - \x'}{|\x - \x'|^3}  = \frac{\partial}{\partial x_i}\int_S d\Omega~,
    }
\end{equation}
which implies that
\begin{equation}
    \oint d\bl' \times \frac{\x - \x'}{|\x - \x'|^3} = \nabla \Omega~.
\end{equation}


\newpage
\noindent{\bf 5.3} In Jackson section 5.5, it already gives the magnetic induction for a circular current loop,
For the position located on the symmetric axis of the loop, the magnetic induction is given by
\begin{equation}
    B_z = \frac{\mu_0 I}{2} \frac{a^2}{(a^2 + z^2)^{3/2}}~,
\end{equation}
where $a$ is the radius of the loop and $z$ is the distance between the center of the loop and the position where we measure the magnetic induction.

In this case, we have
\begin{equation}
    B_z  = \frac{\mu_0 N I}{2} a^2 \int dz  \frac{1}{(a^2 + z^2)^{3/2}}~.
\end{equation}

Instead of using variable $z$, we use the variable $\theta$ such that
\begin{equation}
    \tan \theta = a/z~.
\end{equation}
Then
\begin{equation}
    \boxed{
    B_z = \frac{\mu_0 N I}{2} \int_{\pi - \theta_1}^{\theta_2} d(\cos \theta) = \frac{\mu_0 N I}{2} (\cos \theta_2 + \cos \theta_1)~.
    }
\end{equation}

\newpage
\noindent{\bf 5.4 (a)} In a current-free region, the magnetic induction satisfies
\begin{equation}
    \nabla \cdot \B = 0~, \quad \nabla \times \B = 0~.
\end{equation}
Written in components, we have
\begin{equation}\label{eqn:constraints}
    \frac{1}{\rho} \frac{\partial (\rho B_\rho)}{\partial \rho} + \frac{\partial B_z}{\partial z} = 0~, \quad \frac{\partial B_\rho}{\partial z} - \frac{\partial B_z}{\partial \rho} = 0~.
\end{equation}

Near the axis, the axial component of the magnetic induction can be expanded as
\begin{align}
    B_z(\rho, z) \approx B_z(0,z) + \frac{\partial B_z(0, z)}{\partial \rho} \rho + \frac{1}{2} \frac{\partial^2 B_z(0,z)}{\partial \rho^2} \rho^2 + \frac{1}{6} \frac{\partial^3 B_z(0,z)}{\partial \rho^3} \rho^3+  \cdots
\end{align}
The radial component of the magnetic induction can also be expanded as
\begin{equation}
    B_\rho(\rho, z) \approx B_\rho(0,z) +  \frac{\partial B_\rho(0,z)}{\partial \rho}  \rho + \frac{1}{2} \frac{\partial^2 B_\rho(0,z)}{\partial \rho^2}  \rho^2 + \frac{1}{6} \frac{\partial^3 B_\rho(0,z)}{\partial \rho^3}  \rho^3  + \cdots
\end{equation}
Next we just plug these two expansions into \eqref{eqn:constraints} and equating the coefficients.
\begin{align}
    \frac{\partial B_\rho}{\partial z} &= \frac{\partial B_\rho(0,z)}{\partial z} + \frac{\partial^2 B_\rho(0,z)}{\partial z \partial \rho} \rho + \frac{1}{2} \frac{\partial^3 B_\rho(0,z)}{\partial z \partial \rho^2} \rho^2 + \frac{1}{6} \frac{\partial^4 B_\rho(0,z)}{\partial z \partial \rho^3} \rho^3 + \cdots\\
    \frac{\partial B_z}{\partial \rho} &= \frac{\partial B_z(0,z)}{\partial \rho} + \frac{\partial^2 B_z(0,z)}{\partial \rho^2} \rho + \frac{1}{2} \frac{\partial^3 B_z(0,z)}{\partial \rho^3} \rho^2 + \cdots
\end{align}
and
\begin{align}
    -\frac{\partial B_z}{\partial z} &= - \frac{\partial B_z(0,z)}{\partial z} - \frac{\partial^2 B_z(0,z)}{\partial z \partial \rho} \rho - \frac{1}{2} \frac{\partial^3 B_z(0,z)}{\partial z \partial \rho^2} \rho^2 - \frac{1}{6} \frac{\partial^4 B_z(0,z)}{\partial z \partial \rho^3} \rho^3 + \cdots\\
    \frac{1}{\rho}\frac{\partial (\rho B_\rho)}{\partial \rho} &= \frac{B_\rho(0,z)}{\rho} + 2 \frac{\partial B_\rho(0,z)}{\partial \rho} + \frac{3}{2} \frac{\partial^2 B_\rho(0,z)}{\partial \rho^2} \rho + \frac{2}{3} \frac{\partial^3 B_\rho(0,z)}{\partial \rho^3} \rho^2 + \cdots
\end{align}
We have
\begin{align}
    B_\rho(0,z) & = 0~,\\
    \frac{\partial B_\rho(0,z)}{\partial \rho} &= - \frac{1}{2}\frac{\partial B_z(0,z)}{\partial z}~, \\
    \frac{\partial^2 B_\rho(0,z)}{\partial \rho^2} &= - \frac{2}{3} \frac{\partial^2 B_z(0,z)}{\partial z \partial \rho} = - \frac{2}{3} \frac{\partial^2 B_\rho(0,z)}{\partial z^2} = 0~,\\
    \frac{\partial ^3 B_\rho(0,z)}{\partial \rho^3} &= -\frac{3}{4} \frac{\partial^3 B_z(0,z)}{\partial z \partial \rho^2} = -\frac{3}{4} \frac{\partial^3 B_\rho(0,z)}{\partial z^2 \partial \rho} = \frac{3}{8} \frac{\partial^3 B_z(0,z)}{\partial z^3}~.
\end{align}
We also have
\begin{align}
    \frac{\partial B_z(0,z)}{\partial \rho} &= \frac{\partial B_\rho(0,z)}{\partial z} = 0~,\\
    \frac{\partial^2 B_z(0,z)}{\partial \rho^2} &= \frac{\partial^2 B_\rho(0,z)}{\partial z \partial \rho} = - \frac{1}{2} \frac{\partial^2 B_z(0,z)}{\partial z^2}~,\\
    \frac{\partial^3 B_z(0,z)}{\partial \rho^3} &= \frac{\partial^3 B_\rho(0,z)}{\partial z \partial \rho^2} = -\frac{2}{3}\frac{\partial^3 B_\rho(0,z)}{\partial z^3} = 0~.
\end{align}
Since $B_\rho(0,z) = 0$, all its derivatives with respect to $z$ gives zero.
Therefore,
\begin{empheq}[box=\fbox]{align}
    B_z(\rho, z) &\approx  B_z(0,z) -\frac{1}{4} \frac{\partial^2 B_z(0,z)}{\partial z^2} \rho^2 + \cdots \\
    B_\rho(\rho, z) &\approx -\frac{1}{2} \frac{\partial B_z(0,z)}{\partial z} \rho - \frac{1}{16} \frac{\partial^3 B_z(0,z)}{\partial z^3} \rho^3 + \cdots
\end{empheq}


\newpage
\noindent
{\bf 5.7 (a)} We have
\begin{equation}
    \B = \frac{\mu_0 I}{4\pi} \oint \frac{d\bl' \times (\x - \x')}{|\x - \x'|^3}
\end{equation}
For $\x$ on the z axis, the only non-vanishing component is the $z$ component
\begin{equation}
    B_z(z)= \frac{\mu_0 I}{4\pi} \int_0^{2\pi} d\theta \frac{a^2}{(a^2 + z^2)^{3/2}} = \boxed{\frac{\mu_0 I}{2} \frac{a^2}{(a^2 + z^2)^{3/2}}}~.
\end{equation}

\newpage
\noindent{\bf 5.7 (b)} From (a), we can write down the magnetic induction at the axis as
\begin{equation}
    B_z (z) = \frac{\mu_0 I}{2} \left(\frac{a^2}{(a^2 + (z+ b/2)^2)^{3/2}} + \frac{a^2}{(a^2 + (z- b/2)^2)^{3/2}}\right)~.
\end{equation}
When we look near the origin where $z$ is very small,
\begin{align}\label{eqn:taylor-expansion}
    B_z(z) &= \frac{\mu_0 I a^2}{2} \left(\frac{1}{(a^2+ b^2/4 + z^2 + bz)^{3/2}} + \frac{1}{(a^2 + b^2/4 + z^2 - bz)^{3/2}}\right)\cr
    &= \frac{\mu_0 I a^2}{2 d^3} \left[\left(1 + \frac{z^2 + bz}{d^2}\right)^{-3/2} + \left(1 + \frac{z^2 - bz}{d^2}\right)^{-3/2}\right]
\end{align}
Define
\begin{equation}
    f(z) = \frac{1}{2}\left(1 + \frac{z^2 + bz}{d^2}\right)^{-3/2} + \frac{1}{2}\left(1 + \frac{z^2 - bz}{d^2}\right)^{-3/2}~.
\end{equation}
Now we calculate its Taylor expansion around $z = 0$,
\begin{align}
    f(0) &= 1 \\
    f'(0) &= 0 \\
    f''(0) &= \frac{3(5 b^2 - 4d^2)}{4 d^4} = \frac{3(b^2 - a^2)}{d^4}\\
    f^{(3)} &= 0 \\
    f^{(4)} &= \frac{45(21 b^4 - 56 b^2 d^2 + 16 d^4)}{16 d^8} = \frac{45(b^4 - 6 b^2 a^2+ 2a^4)}{2d^8}
\end{align}
Therefore,
\begin{equation}
    B_z(z) = \left(\frac{\mu_0 I a^2}{d^3}\right) \left[1 + \frac{3(b^2 - a^2) z^2}{2d^4} + \frac{15(b^4 - 6b^2 a^2 + 2a^4)z^4}{16 d^8} + \cdots\right]
\end{equation}

\newpage
\noindent{\bf 5.7(c)} Using results from 5.4 (a), copied here for convenience, for position slightly off the axis, we have
\begin{align}
    B_z(\rho, z) &\approx B_z(0,z) - \left(\frac{\rho^2}{4}\right) \left[\frac{\partial^2 B_z(0,z)}{\partial z^2}\right]+ \cdots\\
    B_\rho(\rho, z) &\approx -\left(\frac{\rho}{2}\right) \left[\frac{\partial B_z(0,z)}{\partial z}\right]
\end{align}
Define
\begin{equation}
    \sigma_0 \equiv \frac{\mu_0 I a^2}{d^3}~, \quad \sigma_2 \equiv \frac{3(b^2-a^2)}{2d^4} \sigma_0
\end{equation}
we have
\begin{equation}
    \frac{\partial B_z(0,z)}{\partial z} = 2\sigma_2 z + \cdots~,\quad
    \frac{\partial^2 B_z(0,z)}{\partial z^2} = 2\sigma_2 + \cdots
\end{equation}
So correct to second order in coordinates
\begin{equation}
    B_z(\rho,z) = \sigma_0 + \sigma_2 \left(z^2 - \frac{\rho^2}{2}\right)~, \quad B_\rho = - \sigma_2 z \rho~.
\end{equation}

\newpage
\noindent{\bf 5.7 (d)} We start from
\begin{align}
    B_z (z) &= \frac{\mu_0 I a^2}{2} \left(\frac{1}{(a^2 + (z+ b/2)^2)^{3/2}} + \frac{1}{(a^2 + (z- b/2)^2)^{3/2}}\right) \cr
    &= \frac{\mu_0 I a^2}{2 |z|^3} \left[\left(1+ bz^{-1} + d^2 z^{-2}\right)^{3/2} + \left(1- bz^{-1} + d^2 z^{-2}\right)^{3/2}\right]
\end{align}
This is the same form as \eqref{eqn:taylor-expansion}, and following the same procedure, we can expand it around large $z$ as
\begin{equation}
    B_z(z)= \frac{\mu_0 I a^2}{|z|^3}\left[1 + \frac{3}{2}\frac{(b^2- a^2)}{z^2} + \frac{15}{16}\frac{(b^4 - 6b^2 a^2 + 2 a^4)}{z^4} + \cdots\right]~.
\end{equation}
It is just obtained from the small $z$ expansion by the formal substitution $d \to |z|$.

\newpage
\noindent{\bf 5.8 (a)} The vector potential is given by
\begin{equation}
    \A(\x) = \frac{\mu_0}{4\pi}\int d^3 x' \frac{{\bf J}(\x')}{|\x - \x'|} = \frac{\mu_0}{4\pi} \int d^3 x' \frac{J(r, \theta) \hat{\phi}'}{|\x - \x'|}~.
\end{equation}
Decompose the spherical coordinates basis in Cartesian coordinates,
\begin{align}
    \hat{r} &= \frac{\partial}{\partial r} = \frac{\partial x}{\partial r} \frac{\partial }{\partial x} + \frac{\partial y}{\partial r} \frac{\partial }{\partial y}+ \frac{\partial z}{\partial r} \frac{\partial }{\partial z} = (\sin \theta \cos \phi, \sin \theta \sin \phi, \cos \theta)\\
    \hat{\bf \theta} &= \frac{1}{r}\frac{\partial}{\partial \theta} = \frac{1}{r}\left(\frac{\partial x}{\partial \theta} \frac{\partial }{\partial x} + \frac{\partial y}{\partial \theta} \frac{\partial }{\partial y}+ \frac{\partial z}{\partial \theta} \frac{\partial }{\partial z}\right) = (\cos \theta \cos \phi, \cos \theta \sin \phi, -\sin \theta)\\
    \hat{\phi} &= \frac{1}{r\sin\theta} \frac{\partial}{\partial \phi} =\frac{1}{r \sin \theta}\left(\frac{\partial x}{\partial \phi} \frac{\partial }{\partial x} + \frac{\partial y}{\partial \phi} \frac{\partial }{\partial y}+ \frac{\partial z}{\partial \phi} \frac{\partial }{\partial z}\right) = (-\sin \phi, \cos \phi, 0)
\end{align}
Therefore,
\begin{equation}
    \hat{\phi}' = (\hat{\phi}' \cdot \hat{r} )\hat{r} + (\hat{\phi}' \cdot \hat{\theta})\hat{\theta} + (\hat{\phi}' \cdot \hat{\phi}) \hat{\phi} = \sin \theta \sin(\phi - \phi') \hat{r} + \cos\theta \sin(\phi - \phi')\hat{\theta} + \cos(\phi - \phi') \hat{\phi}~.
\end{equation}
Moreover, we can expand the Green's function in terms of spherical harmonics,
\begin{equation}
    \A = \frac{\mu_0}{4\pi} \int r'^2 dr' d\Omega' \hat{\phi}'J(r', \theta') \sum_{L}\sum_{m} \frac{4\pi}{2L+1}\left(\frac{r_<^L}{r_>^{L+1}}\right) Y_{Lm}^*(\theta', \phi') Y_{L,m}(\theta, \phi)
\end{equation}
For $m = 0$, the integral over $\phi'$ vanishes. For $m \neq 0$,
since
\begin{align}
    Y_{Lm}^*(\theta', \phi')Y_{Lm}(\theta, \phi) &= Y_{Lm}(\theta',0)Y_{Lm}(\theta,0)e^{im(\phi - \phi')}\\
    Y_{L,-m}^*(\theta', \phi')Y_{L,-m}(\theta, \phi) &= Y_{Lm}(\theta',0)Y_{Lm}(\theta,0)e^{-im(\phi - \phi')}~,
\end{align}
so
\begin{equation}
    \A = \frac{\mu_0}{2\pi} \int r'^2 dr' d\Omega' \hat{\phi}'J(r', \theta') \sum_{L}\sum_{m=1}^\infty \frac{4\pi}{2L+1}\left(\frac{r_<^L}{r_>^{L+1}}\right) Y_{Lm}^*(\theta', 0) Y_{L,m}(\theta, 0)\cos(m(\phi - \phi'))
\end{equation}
Performing the $\phi'$ integral, the only non-vanishing contribution is the $\hat{\phi}$ component with $m = 1$, therefore,
\begin{equation}
    \A = \hat{\phi} A_\phi = \hat{\phi} \frac{\mu_0}{2} \int r'^2 dr' \sin\theta' d\theta' J(r', \theta') \sum_{L} \frac{1}{L(L+1)} \left(\frac{r_<^L}{r_>^{L+1}}\right) P_L^1(\cos \theta) P_L^1(\cos \theta')~.
\end{equation}
In the interior, $r_< = r$ and $r_> = r'$, so we have
\begin{equation}
    A_\phi = -\frac{\mu_0}{4\pi} \sum_L m_L r^L P_L^1(\cos \theta)~,
\end{equation}
where
\begin{align}
    m_L &= -2\pi \frac{1}{L(L+1)}\int r'^2 dr' \sin \theta' d \theta' J(r', \theta') r'^{-L-1} P_L^1(\cos \theta') \cr
    &= -\frac{1}{L(L+1)} \int d^3 x r^{-L-1} P_L^1(\cos \theta) J(r, \theta)~.
\end{align}

Similarly, outside the current distribution, $r_< = r'$, $r_> = r$, so we have
\begin{equation}
    A_\phi (r, \theta) = -\frac{\mu_0}{4\pi} \sum_{L} \mu_L r^{-L-1} P_L^1(\cos \theta)~,
\end{equation}
where
\begin{align}
    \mu_L &= -2\pi \frac{1}{L(L+1)} \int r'^2 dr' \sin \theta' d\theta' J(r', \theta') r'^L P_L^1(\cos\theta')\cr
    &= - \frac{1}{L(L+1)} \int d^3 x r^L P_L^1(\cos \theta) J(r, \theta)~.
\end{align}

\newpage
\noindent{\bf 5.10 (a)}
We have to show that
\begin{equation}
    A_\phi (\rho, z ) = \frac{\mu_0 I a}{\pi} \int_0^\infty dk \cos kz  I_1(k \rho_<) K_1(k \rho_>)~.
\end{equation}
The current density is given by
\begin{equation}
    {\bf J}(\x') = \hat{\phi}' I \delta(\rho'-a) \delta(z')~.
\end{equation}
As we have derived in the last problem,
\begin{equation}
    \hat{\phi}' = \sin \theta \sin(\phi - \phi') \hat{r} + \cos\theta \sin(\phi - \phi')\hat{\theta} + \cos(\phi - \phi') \hat{\phi}~.
\end{equation}
Using Jackson (3.149),
\begin{equation}\nonumber
    \frac{1}{|\x - \x'|} = \frac{4}{\pi} \int_0^\infty dk \cos[k(z-z')]
    \left(
        \frac{1}{2} I_0(k \rho_<) K_0(k \rho_>) + \sum_{m=1}^\infty  \cos[m(\phi - \phi')] I_m (k \rho_<) K_m(k \rho_>)~,
    \right)
\end{equation}
where $\rho_< = \min(\rho, \rho')$ and $\rho_> = \max(\rho, \rho')$.
Therefore, from
\begin{align}
    \A(\x)&= \frac{\mu_0}{4\pi} \int d^3 x' \frac{{\bf J}(\x')}{|\x - \x'|}~,
\end{align}
we can see that for $\hat{r}$ and $\hat{\theta}$ component, the $\theta$ integral vanishes and the only non-vanishing component is
\begin{align}
    A_\phi(\rho,z) =&~ \frac{\mu_0 I }{\pi^2} \int d^3 x' \int dk \cos(\phi - \phi') \delta(\rho' - a) \delta(z') \cos[k(z-z')]\cr
    &\times\left(
        \frac{1}{2} I_0(k \rho_<) K_0(k \rho_>) + \sum_{m=1}^\infty  \cos[m(\phi - \phi')] I_m (k \rho_<) K_m(k \rho_>)
    \right)\cr
    =&~ \frac{\mu_0 I a}{\pi} \int_0^\infty dk \cos(kz) I_1(k \rho_<) K_1(k \rho_>)~,
\end{align}
where in the last step, only $m=1$ term survives and $\rho_<  = \min(a, \rho)$ and $\rho_> = \max(a, \rho)$.

\newpage
\noindent{\bf 5.10 (b)} We want to show that an alternative expression for $A_\phi$ is
\begin{equation}
    A_\phi(\rho, z) = \frac{\mu_0 I a}{2} \int_0^\infty dk e^{-k|z|} J_1(ka) J_1(k \rho)~.
\end{equation}
From Jackson problem 3.16 (b), we have
\begin{equation}
    \frac{1}{|\x - \x'|} = \sum_{m = - \infty}^\infty \int_0^\infty dk e^{im(\phi - \phi')} J_m(k \rho) J_m(k \rho') e^{-k(z_> - z_<)}~.
\end{equation}
Therefore,
\begin{align}
    A_\phi(\rho,z) &= \frac{\mu_0 I}{4\pi} \int d^3 x'\int dk  \delta(\rho' - a) \delta(z') \cos(\phi - \phi') \sum_{m = -\infty}^\infty e^{im(\phi - \phi')} J_m(k\rho) J_m(k \rho') e^{-k(z_> - z_<)}\cr
    &= \frac{\mu_0 I a}{4\pi} \int_0^{2\pi} d\phi' \int_0^\infty dk \cos(\phi - \phi') \sum_{m = -\infty}^\infty e^{im(\phi - \phi')} J_m(k \rho) J_m(k a) e^{-k|z|}
\end{align}
Since $J_{-m}(x) = (-1)^m J_m(x)$
\begin{equation}
    \sum_{m = -\infty}^\infty e^{im(\phi - \phi')} J_m(k \rho) J_m(k a) = J_0(k \rho) J_0 (ka) + 2 \sum_{m = 1}^\infty J_m(k \rho) J_m(ka) \cos(\phi - \phi')~.
\end{equation}
Only the $m = 1$ term contributes to the integral, so
\begin{equation}
    A_\phi(\rho,z) = \frac{\mu_0 I a}{2} \int_0^\infty e^{-k|z|} J_1(k \rho) J_1(k a)~.
\end{equation}


\end{document}