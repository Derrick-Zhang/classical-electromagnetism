\documentclass[12pt]{article}
\usepackage{fullpage}

\usepackage{amsmath,amssymb}
\usepackage{hyperref}
\usepackage{empheq}

\newcommand{\Z}{\mathbb{Z}}
\newcommand{\x}{{\bf x}}
\newcommand{\p}{{\bf p}}
\newcommand{\lb}{\left(}
\newcommand{\rb}{\right)}
\newcommand{\mb}{{m \pi/\beta}}
\newcommand{\B}{{\bf B}}
\newcommand{\bl}{{\bf l}}

\begin{document}
\begin{center}
{\bf Phys 5405}\\
HW 10 \\
5.1, 5.3, 5.4(a), 5.7(a,b,c,d), 5.8, 5.10(a,b)

\end{center}
\noindent{\bf 5.1}
We have the differential expression
\begin{equation}
    d {\bf B} = \frac{\mu_0 I}{4\pi}\, d\bl' \times \frac{\x - \x'}{|\x - \x'|^3}~.
\end{equation}
Then
\begin{equation}
    \B = \int d \B = \frac{\mu_0 I}{4\pi} \oint d \bl' \times \frac{\x - \x'}{|\x - \x'|^3}~.
\end{equation}
It suffices to show that
\begin{equation}
    \oint d\bl' \times \frac{\x - \x'}{|\x - \x'|^3} = \nabla \Omega~.
\end{equation}

First, we can make use of the identity,
\begin{equation}
    \boxed{
    \frac{\x - \x'}{|\x - \x'|^3} = \nabla' \left(\frac{1}{|\x - \x'|}\right) \equiv {\bf V}~.
    }
\end{equation}
Then, consider each components
\begin{align}
    \boxed{
    \hat{\x}_i \cdot \oint d\bl' \times {\bf V} = \oint d\bl' \cdot ({\bf V} \times \hat{\x}_i) = \int_S d{\bf a}' \cdot (\nabla' \times({\bf V} \times \hat{\x}_i))~.
    }
\end{align}
Now we evaluate $\nabla' \times({\bf V} \times \hat{\x}_i)$, using the identity
\begin{equation}
    \nabla \times ({\bf A} \times {\bf B}) = {\bf A}(\nabla \cdot {\bf B}) - {\bf B}(\nabla \cdot {\bf A}) + ({\bf B} \cdot \nabla) {\bf A} - ({\bf A} \cdot \nabla) {\bf B}~,
\end{equation}
so we have
\begin{equation}
    \boxed{
    \nabla' \times({\bf V} \times \hat{\x}_i) = - \hat{\x}_i(\nabla' \cdot {\bf V}) + (\hat{\x}_i \cdot \nabla') {\bf V} = (\hat{\x}_i \cdot \nabla') {\bf V} = \frac{\partial}{\partial x'_i} {\bf V}~,
    }
\end{equation}
where we have used the fact that for $\x \neq \x'$,
\begin{equation}
    \nabla' \cdot {\bf V} = \nabla'^2 \left(\frac{1}{|\x - \x'|} \right) = 0~.
\end{equation}
Finally, we have achieved that
\begin{equation}
    \hat{\x}_i \cdot \oint d\bl' \times \frac{\x - \x'}{|\x - \x'|^3} = \int_S d{\bf a}' \cdot \frac{\partial}{\partial x_i'} \left(\nabla' \left(\frac{1}{|\x - \x'|}\right)\right) = -\frac{\partial}{\partial x_i}\int_S d{\bf a}' \cdot \nabla' \left(\frac{1}{|\x - \x'|}\right)
\end{equation}
From Jackson, the equation below (1.25), we have
\begin{equation}
    d{\bf a}' \cdot \nabla' \left(\frac{1}{|\x - \x'|}\right) = - d\Omega~.
\end{equation}
Therefore,
\begin{equation}
    \boxed{
    \hat{\x}_i \cdot \oint d\bl' \times \frac{\x - \x'}{|\x - \x'|^3}  = \frac{\partial}{\partial x_i}\int_S d\Omega~,
    }
\end{equation}
which implies that
\begin{equation}
    \oint d\bl' \times \frac{\x - \x'}{|\x - \x'|^3} = \nabla \Omega~.
\end{equation}


\newpage
\noindent{\bf 5.3} In Jackson section 5.5, it already gives the magnetic induction for a circular current loop,
For the position located on the symmetric axis of the loop, the magnetic induction is given by
\begin{equation}
    B_z = \frac{\mu_0 I}{2} \frac{a^2}{(a^2 + z^2)^{3/2}}~,
\end{equation}
where $a$ is the radius of the loop and $z$ is the distance between the center of the loop and the position where we measure the magnetic induction.

In this case, we have
\begin{equation}
    B_z  = \frac{\mu_0 N I}{2} a^2 \int dz  \frac{1}{(a^2 + z^2)^{3/2}}~.
\end{equation}

Instead of using variable $z$, we use the variable $\theta$ such that
\begin{equation}
    \tan \theta = a/z~.
\end{equation}
Then
\begin{equation}
    \boxed{
    B_z = \frac{\mu_0 N I}{2} \int_{\pi - \theta_1}^{\theta_2} d(\cos \theta) = \frac{\mu_0 N I}{2} (\cos \theta_2 + \cos \theta_1)~.
    }
\end{equation}

\newpage
\noindent{\bf 5.4 (a)} In a current-free region, the magnetic induction satisfies
\begin{equation}
    \nabla \cdot \B = 0~, \quad \nabla \times \B = 0~.
\end{equation}
Written in components, we have
\begin{equation}\label{eqn:constraints}
    \frac{1}{\rho} \frac{\partial (\rho B_\rho)}{\partial \rho} + \frac{\partial B_z}{\partial z} = 0~, \quad \frac{\partial B_\rho}{\partial z} - \frac{\partial B_z}{\partial \rho} = 0~.
\end{equation}

Near the axis, the axial component of the magnetic induction can be expanded as
\begin{align}
    B_z(\rho, z) \approx B_z(0,z) + \frac{\partial B_z(0, z)}{\partial \rho} \rho + \frac{1}{2} \frac{\partial^2 B_z(0,z)}{\partial \rho^2} \rho^2 + \frac{1}{6} \frac{\partial^3 B_z(0,z)}{\partial \rho^3} \rho^3+  \cdots
\end{align}
The radial component of the magnetic induction can also be expanded as
\begin{equation}
    B_\rho(\rho, z) \approx B_\rho(0,z) +  \frac{\partial B_\rho(0,z)}{\partial \rho}  \rho + \frac{1}{2} \frac{\partial^2 B_\rho(0,z)}{\partial \rho^2}  \rho^2 + \frac{1}{6} \frac{\partial^3 B_\rho(0,z)}{\partial \rho^3}  \rho^3  + \cdots
\end{equation}
Next we just plug these two expansions into \eqref{eqn:constraints} and equating the coefficients.
\begin{align}
    \frac{\partial B_\rho}{\partial z} &= \frac{\partial B_\rho(0,z)}{\partial z} + \frac{\partial^2 B_\rho(0,z)}{\partial z \partial \rho} \rho + \frac{1}{2} \frac{\partial^3 B_\rho(0,z)}{\partial z \partial \rho^2} \rho^2 + \frac{1}{6} \frac{\partial^4 B_\rho(0,z)}{\partial z \partial \rho^3} \rho^3 + \cdots\\
    \frac{\partial B_z}{\partial \rho} &= \frac{\partial B_z(0,z)}{\partial \rho} + \frac{\partial^2 B_z(0,z)}{\partial \rho^2} \rho + \frac{1}{2} \frac{\partial^3 B_z(0,z)}{\partial \rho^3} \rho^2 + \cdots
\end{align}
and
\begin{align}
    -\frac{\partial B_z}{\partial z} &= - \frac{\partial B_z(0,z)}{\partial z} - \frac{\partial^2 B_z(0,z)}{\partial z \partial \rho} \rho - \frac{1}{2} \frac{\partial^3 B_z(0,z)}{\partial z \partial \rho^2} \rho^2 - \frac{1}{6} \frac{\partial^4 B_z(0,z)}{\partial z \partial \rho^3} \rho^3 + \cdots\\
    \frac{1}{\rho}\frac{\partial (\rho B_\rho)}{\partial \rho} &= \frac{B_\rho(0,z)}{\rho} + 2 \frac{\partial B_\rho(0,z)}{\partial \rho} + \frac{3}{2} \frac{\partial^2 B_\rho(0,z)}{\partial \rho^2} \rho + \frac{2}{3} \frac{\partial^3 B_\rho(0,z)}{\partial \rho^3} \rho^2 + \cdots
\end{align}
We have
\begin{align}
    B_\rho(0,z) & = 0~,\\
    \frac{\partial B_\rho(0,z)}{\partial \rho} &= - \frac{1}{2}\frac{\partial B_z(0,z)}{\partial z}~, \\
    \frac{\partial^2 B_\rho(0,z)}{\partial \rho^2} &= - \frac{2}{3} \frac{\partial^2 B_z(0,z)}{\partial z \partial \rho} = - \frac{2}{3} \frac{\partial^2 B_\rho(0,z)}{\partial z^2} = 0~,\\
    \frac{\partial ^3 B_\rho(0,z)}{\partial \rho^3} &= -\frac{3}{4} \frac{\partial^3 B_z(0,z)}{\partial z \partial \rho^2} = -\frac{3}{4} \frac{\partial^3 B_\rho(0,z)}{\partial z^2 \partial \rho} = \frac{3}{8} \frac{\partial^3 B_z(0,z)}{\partial z^3}~.
\end{align}
We also have
\begin{align}
    \frac{\partial B_z(0,z)}{\partial \rho} &= \frac{\partial B_\rho(0,z)}{\partial z} = 0~,\\
    \frac{\partial^2 B_z(0,z)}{\partial \rho^2} &= \frac{\partial^2 B_\rho(0,z)}{\partial z \partial \rho} = - \frac{1}{2} \frac{\partial^2 B_z(0,z)}{\partial z^2}~,\\
    \frac{\partial^3 B_z(0,z)}{\partial \rho^3} &= \frac{\partial^3 B_\rho(0,z)}{\partial z \partial \rho^2} = -\frac{2}{3}\frac{\partial^3 B_\rho(0,z)}{\partial z^3} = 0~.
\end{align}
Since $B_\rho(0,z) = 0$, all its derivatives with respect to $z$ gives zero.
Therefore,
\begin{empheq}[box=\fbox]{align}
    B_z(\rho, z) &\approx  B_z(0,z) -\frac{1}{4} \frac{\partial^2 B_z(0,z)}{\partial z^2} \rho^2 + \cdots \\
    B_\rho(\rho, z) &\approx -\frac{1}{2} \frac{\partial B_z(0,z)}{\partial z} \rho - \frac{1}{16} \frac{\partial^3 B_z(0,z)}{\partial z^3} \rho^3 + \cdots
\end{empheq}


\newpage
\noindent
{\bf 5.7 (a)} We have

\end{document}