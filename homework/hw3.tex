\documentclass[12pt]{article}

\usepackage{amsmath,amssymb}
\usepackage{hyperref}

\newcommand{\Z}{\mathbb{Z}}

\begin{document}

\begin{center}
{\bf Phys 5405}\\
HW 3
\end{center}
\textbf{2.3} (a) By symmetry, we can simply write down the answer,
\begin{align}
    \Phi(x,y) = &~ \frac{\lambda}{4 \pi \epsilon_0} \ln\frac{R^2}{(x-x_0)^2 + (y - y_0)^2}
    - \frac{\lambda}{4 \pi \epsilon_0} \ln\frac{R^2}{(x+x_0)^2 + (y - y_0)^2}\cr
    &- \frac{\lambda}{4 \pi \epsilon_0} \ln\frac{R^2}{(x-x_0)^2 + (y + y_0)^2} + \frac{\lambda}{4 \pi \epsilon_0} \ln\frac{R^2}{(x+x_0)^2 + (y + y_0)^2}~.\nonumber
\end{align}
Cancelling the constant terms, we obtain
\begin{align}
    \Phi(x,y) = &~ -\frac{\lambda}{4 \pi \epsilon_0} \ln\left({(x-x_0)^2 + (y - y_0)^2}\right)
    + \frac{\lambda}{4 \pi \epsilon_0} \ln\left({(x+x_0)^2 + (y - y_0)^2}\right)\cr
    &+ \frac{\lambda}{4 \pi \epsilon_0} \ln\left({(x-x_0)^2 + (y + y_0)^2}\right) - \frac{\lambda}{4 \pi \epsilon_0} \ln\left({(x+x_0)^2 + (y + y_0)^2}\right)~.\nonumber
\end{align}
It is easily verified that the potential vanishes on the boundary. Now consider the tangential electric field. At $x = 0$, the tangential electric field should be along $y$ axis.
\begin{align} \label{eqn:partial-y}
    E_y(x,y) =&~ - \partial_y \Phi(x,y) \cr
    =&~ \frac{\lambda}{4\pi\epsilon_0} \frac{2(y-y_0)}{(x-x_0)^2 + (y - y_0)^2} - \frac{\lambda}{4\pi\epsilon_0} \frac{2(y-y_0)}{(x+x_0)^2 + (y - y_0)^2}\cr
    &-\frac{\lambda}{4\pi\epsilon_0} \frac{2(y+y_0)}{(x-x_0)^2 + (y + y_0)^2} + \frac{\lambda}{4\pi\epsilon_0} \frac{2(y+y_0)}{(x+x_0)^2 + (y + y_0)^2}
\end{align}
It is easily seen that at $x = 0$,
\begin{equation}
    E_y (x=0,y) = 0~.
\end{equation}
Similarly, at $y = 0$, the tangential electric field should be along $x$ axis.
\begin{align} \label{eqn:partial-x}
    E_x(x,y) =&~ - \partial_x \Phi(x,y) \cr
    =&~ \frac{\lambda}{4\pi\epsilon_0} \frac{2(x-x_0)}{(x-x_0)^2 + (y - y_0)^2} - \frac{\lambda}{4\pi\epsilon_0} \frac{2(x+x_0)}{(x+x_0)^2 + (y - y_0)^2}\cr
    &-\frac{\lambda}{4\pi\epsilon_0} \frac{2(x-x_0)}{(x-x_0)^2 + (y + y_0)^2} + \frac{\lambda}{4\pi\epsilon_0} \frac{2(x+x_0)}{(x+x_0)^2 + (y + y_0)^2}
\end{align}
It is easily seen that at $y = 0$,
\begin{equation}
    E_x(x,y=0) = 0~.
\end{equation}

\newpage
\noindent {\bf 2.3} (c) At $y = 0$, the surface charge density is
\begin{equation}
    \sigma(x) = - \epsilon_0 \frac{\partial \Phi}{\partial y}\Bigg|_{y = 0}
\end{equation}
Now using result in \eqref{eqn:partial-y}, we have
\begin{equation}
    \sigma(x) = -\frac{\lambda}{\pi}\left( \frac{y_0}{(x-x_0)^2 + y_0^2} - \frac{y_0}{(x+x_0)^2 + y_0^2} \right)~.
\end{equation}
Then the total charge, per unit length in $z$ on the plane $y = 0, x \ge 0$ is
\begin{align}
    Q_x &= \int_0^\infty dx\, \sigma(x) = -\frac{\lambda}{\pi}\int_0^\infty dx\left( \frac{y_0}{(x-x_0)^2 + y_0^2} - \frac{y_0}{(x+x_0)^2 + y_0^2} \right)\cr
    &= -\frac{\lambda}{\pi} \left(\tan^{-1}\left(\frac{x-x_0}{y_0}\right)\Bigg|_{0}^{\infty} - \tan^{-1}\left(\frac{x+x_0}{y_0}\right)\Bigg|_{0}^{\infty} \right)\cr
    &= - \frac{2}{\pi} \lambda \tan^{-1} \left(\frac{x_0}{y_0}\right)~.
\end{align}

\newpage
\noindent{\bf 2.3} (d) At position far from the origin $\rho \gg \rho_0$, where
\begin{equation}
    \rho = \sqrt{x^2 + y^2}, \quad \rho_0 = \sqrt{x_0^2 + y_0^2}
\end{equation}
We can expand the potential
\begin{align}
    \Phi(x,y) = &~ -\frac{\lambda}{4 \pi \epsilon_0} \ln\left({(x-x_0)^2 + (y - y_0)^2}\right)
    + \frac{\lambda}{4 \pi \epsilon_0} \ln\left({(x+x_0)^2 + (y - y_0)^2}\right)\cr
    &+ \frac{\lambda}{4 \pi \epsilon_0} \ln\left({(x-x_0)^2 + (y + y_0)^2}\right) - \frac{\lambda}{4 \pi \epsilon_0} \ln\left({(x+x_0)^2 + (y + y_0)^2}\right)~.\cr
    =&~ -\frac{\lambda}{4 \pi \epsilon_0} \ln\left(1 + \frac{\rho_0^2 - 2x_0 x - 2y_0 y}{\rho^2}\right) + \frac{\lambda}{4\pi \epsilon_0} \ln\left(1 + \frac{\rho_0^2 + 2x_0x - 2y_0y}{\rho^2}\right)\cr
    &+ \frac{\lambda}{4\pi \epsilon_0} \ln\left(1 + \frac{\rho_0^2 - 2x_0x + 2y_0 y}{\rho^2} \right) - \frac{\lambda}{4\pi \epsilon_0}\ln\left( 1 + \frac{\rho_0^2 + 2x_0x + 2y_0y}{\rho^2}\right)~.\nonumber
\end{align}
Now using $\ln(1 + x) \approx x - \frac{x^2}{2}$ when $x$ is very small, we have
\begin{align}
    \Phi(x,y) \approx -\frac{\lambda}{4\pi \epsilon_0} \Bigg(&\frac{\rho_0^2 - 2x_0 x - 2y_0y}{\rho^2} -\frac{\rho_0^2 + 2x_0x -2y_0y}{\rho^2}  \cr
    &-\frac{\rho_0^2 - 2x_0 x + 2y_0 y}{\rho^2} + \frac{\rho_0^2 + 2x_0 x + 2y_0 y}{\rho^2}\Bigg)\cr
    + \frac{\lambda}{8\pi \epsilon_0} \Bigg[& \left(\frac{\rho_0^2 - 2x_0 x - 2y_0y}{\rho^2}\right)^2 - \left(\frac{\rho_0^2 + 2x_0 x - 2y_0y}{\rho^2}\right)^2 \cr
    &- \left(\frac{\rho_0^2 - 2x_0 x + 2y_0 y}{\rho^2} \right)^2 + \left(\frac{\rho_0^2 + 2x_0 x + 2y_0y}{\rho^2} \right)^2 \Bigg]\cr
    =\frac{\lambda}{8 \pi \epsilon_0 \rho^4} &\times 32 x_0 y_0 x y = \frac{4\lambda}{\pi \epsilon_0} \frac{(x_0 y_0)(xy)}{\rho^4}~.
\end{align}

\newpage
\noindent{\bf 2.7} (a) The general Green function is of the form
\begin{equation}
    G(\vec x, \vec x') = \frac{1}{|\vec x - \vec x'|} + F(\vec x, \vec x')~,
\end{equation}
where $\nabla^2 F(\vec x, \vec x') = 0$. And the Dirichlet boundary condition implies that the Green function has to vanish on $z = 0$ surface. We can view the problem as a unit charge at $x'$, with a conducting plane at $z = 0$. Suppose $\vec x' = (x', y', z')$, we can then put the image charge at $(x',y',-z')$ with negative unit charge. The potential is then given by
\begin{align}
    \Phi(\vec x) = \frac{q}{4\pi \epsilon_0} \Bigg(\frac{1}{\sqrt{(x-x')^2 + (y-y')^2 + (z-z')^2}} \cr
    - \frac{1}{\sqrt{(x-x')^2 + (y-y')^2 + (z+z')^2}}\Bigg)~.
\end{align}
And the Green function should be
\begin{align}
    G_D(\vec x, \vec x') =  \frac{1}{|\vec x - \vec x'|} - \frac{1}{|\vec x - \vec x''|}~,
\end{align}
where $\vec x'' = (x', y', -z')$.

\newpage
\noindent{\bf 2.7} (b) We have
\begin{equation}
    \Phi(\vec x) = \frac{1}{4\pi \epsilon_0} \int_V \rho(\vec x') G_D(\vec x, \vec x') d^3 x' - \frac{1}{4\pi} \oint_S \Phi(\vec x') \frac{\partial G_D}{\partial n'} da'
\end{equation}
Since there is no charge distribution, we have
\begin{equation}
    \Phi(\vec x) = - \frac{1}{4\pi} \oint_S \Phi(\vec x') \frac{\partial G_D}{\partial n'} da'~.
\end{equation}
The normal vector direction is along the $-z'$ direction. So
\begin{equation}
    \frac{\partial G_D}{\partial n'} = - \frac{\partial G_D}{\partial z'} = -\frac{z-z'}{|\vec x - \vec x'|^3} - \frac{z + z'}{|\vec x - \vec x''|^3}~.
\end{equation}
When restricted to the surface, we have $\vec x' = \vec x'' = (x', y', 0)$ and
\begin{equation}
    \frac{\partial G_D}{\partial n'} = - \frac{2 z}{|\vec x - \vec x'|^3}~.
\end{equation}
Now we change to cylindrical coordinates with
$$
x = \rho \cos \phi, \quad y = \rho \sin \phi, \quad x' = \rho' \cos \phi', \quad y' = \rho' \sin \phi'~.
$$
Then in cylindrical coordinates,
\begin{equation}
    \frac{\partial G_D}{\partial n'} = -\frac{2z}{(\rho^2 + \rho'^2 - 2\rho \rho' \cos(\phi - \phi') + z^2)^{3/2}}
\end{equation}
and
\begin{align}
    \Phi(\rho, \phi, \theta) = - \frac{V}{4\pi} \int_0^a d\rho' \int_0^{2\pi} \rho' d \phi' \, \frac{\partial G_D}{\partial n'}
\end{align}
After simplification,
\begin{equation} \label{eqn:phi}
    \Phi(\rho, \phi, \theta) =  \frac{Vz}{2\pi} \int_0^a d\rho' \int_0^{2\pi} \rho' d \phi' \, (\rho ^2 + \rho'^2 - 2\rho \rho' \cos(\phi - \phi') + z^2)^{-3/2}~.
\end{equation}

\newpage
\noindent{\bf 2.7} (c) Using \eqref{eqn:phi}, at $\rho = 0$, we have
\begin{align}
    \Phi &= \frac{Vz}{2\pi} \int_0^a d\rho' \int_0^{2\pi} \rho' d \phi' \, ( \rho'^2 + z^2)^{-3/2}\cr
    &= Vz \int_0^a d\rho' \frac{\rho'}{(\rho'^2 + z^2)^{3/2}} \cr
    & = - Vz \int_0^a d \rho' \frac{\partial}{\partial \rho'} \left(\frac{1}{\sqrt{\rho'^2 + z^2} }\right)\cr
    & = V \left(1 - \frac{z}{\sqrt{a^2 + z^2}}\right)~.
\end{align}

\newpage
\noindent{\bf 2.7} (d) For $\rho^2 + z^2 \gg a^2$, we have
\begin{align}
    \Phi &=  \frac{Vz}{2\pi} \int_0^a d\rho' \int_0^{2\pi} \rho' d \phi' \, (\rho ^2 + \rho'^2 - 2\rho \rho' \cos(\phi - \phi') + z^2)^{-3/2}\cr
    &=\frac{Vz}{2\pi} \frac{1}{(\rho^2 + z^2)^{3/2}} \int_0^a d\rho'\int_0^{2\pi} \rho' d \phi' \, \left(1+ \frac{\rho'^2 - 2\rho \rho' \cos(\phi - \phi')}{\rho^2 + z^2}\right)^{-3/2} \cr
    & = \frac{Vz}{2\pi} \frac{1}{(\rho^2 + z^2)^{3/2}} \int_0^a d\rho' \int_0^{2\pi} \rho' d\phi' \left(1 - \frac32 u + \frac{15}{8} u^2+ \cdots\right)~,
\end{align}
where
\begin{equation}
    u = \frac{\rho'^2 - 2\rho \rho' \cos(\phi - \phi')}{\rho^2 + z^2}~.
\end{equation}
Evaluating the integrals terms by terms
\begin{align}
    &\int_0^a d \rho' \int_0^{2\pi} \rho' d \phi'\, 1 = \pi a^2 \cr
    &\int_0^a d \rho' \int_0^{2\pi} \rho' d \phi'\, \left(-\frac32 u\right)= -\frac32 \frac{1}{\rho^2 + z^2} \left(\frac{\pi}{2}a^4\right)\cr
    &\int_0^a d \rho' \int_0^{2\pi} \rho' d \phi'\, \left(\frac{15}{8} u^2\right)= \frac{15}{8} \frac{1}{(\rho^2 + z^2)^2}  2\pi \left( \frac16 a^6 + \frac 12 \rho^2 a^4\right)~.
\end{align}
Collecting these terms, we have
\begin{equation} \label{eqn:phi2}
    \Phi = \frac{Va^2}{2} \frac{z}{(\rho^2 + z^2)^{3/2}} \left[1 - \frac{3a^2}{4(\rho^2 + z^2)} + \frac{5(3\rho^2 a^2 + a^4)}{8(\rho^2 + z^2)^2} + \cdots\right]
\end{equation}
Check: from \eqref{eqn:phi2}, when $\rho = 0$, we have
\begin{equation} \label{eqn:check}
    \Phi = \frac{Va^2}{2z} \left( 1- \frac34 \frac{a^2}{z^2} + \frac{5}{8} \frac{a^4}{z^4} + \cdots \right)~.
\end{equation}
From \eqref{eqn:phi}, when $z \gg a$, the potential
\begin{equation} \label{eqn:check2}
\Phi = V\left(1 - \frac{1}{\sqrt{1 + \frac{a^2}{z^2}}}\right)~.
\end{equation}
Since for small $u$,
\begin{equation}
    \frac{1}{\sqrt{1+u}} = 1 - \frac{1}{2}u + \frac{3}{8}u^2 - \frac{5}{16} x^3 + \cdots
\end{equation}
Therefore, we can expand \eqref{eqn:check2} as
\begin{align}
    \Phi &= V\left(1 - 1 + \frac12 \frac{a^2}{z^2}  - \frac38 \frac{a^4}{z^4}  + \frac{5}{16} \frac{a^6}{z^6} +\cdots\right) \cr
    &=\frac{Va^2}{2z^2} \left( 1- \frac34 \frac{a^2}{z^2} + \frac{5}{8} \frac{a^4}{z^4} + \cdots\right),
\end{align}
in agreement with \eqref{eqn:check}.
\end{document}