\documentclass[12pt]{article}

\usepackage{amsmath,amssymb}
\usepackage{hyperref}

\newcommand{\Z}{\mathbb{Z}}

\begin{document}

\begin{center}
{\bf Phys 5405}\\
HW 3
\end{center}
\textbf{2.3} (a) Using method of image, substitute the intersecting planes with a straight-line charge with constant linear charge density $\lambda'$, located perpendicular to the $x-y$ plane at point $(a,b)$. Since the system is translation invariant along $z$ axis. We write the potential as a function of $x$ and $y$. By symmetry, we can simply write down the answer,
\begin{align}
    \Phi(x,y) = &~ \frac{\lambda}{4 \pi \epsilon_0} \ln\frac{R^2}{(x-x_0)^2 + (y - y_0)^2}
    - \frac{\lambda}{4 \pi \epsilon_0} \ln\frac{R^2}{(x+x_0)^2 + (y - y_0)^2}\cr
    &- \frac{\lambda}{4 \pi \epsilon_0} \ln\frac{R^2}{(x-x_0)^2 + (y + y_0)^2} + \frac{\lambda}{4 \pi \epsilon_0} \ln\frac{R^2}{(x+x_0)^2 + (y + y_0)^2}~.\nonumber
\end{align}
Cancelling the constant terms, we obtain
\begin{align}
    \Phi(x,y) = &~ -\frac{\lambda}{4 \pi \epsilon_0} \ln\left({(x-x_0)^2 + (y - y_0)^2}\right)
    + \frac{\lambda}{4 \pi \epsilon_0} \ln\left({(x+x_0)^2 + (y - y_0)^2}\right)\cr
    &+ \frac{\lambda}{4 \pi \epsilon_0} \ln\left({(x-x_0)^2 + (y + y_0)^2}\right) - \frac{\lambda}{4 \pi \epsilon_0} \ln\left({(x+x_0)^2 + (y + y_0)^2}\right)~.\nonumber
\end{align}
It is easily verified that the potential vanishes on the boundary. Now consider the tangential electric field. At $x = 0$, the tangential electric field should be along $y$ axis.
\begin{align} \label{eqn:partial-y}
    E_y(x,y) =&~ - \partial_y \Phi(x,y) \cr
    =&~ \frac{\lambda}{4\pi\epsilon_0} \frac{2(y-y_0)}{(x-x_0)^2 + (y - y_0)^2} - \frac{\lambda}{4\pi\epsilon_0} \frac{2(y-y_0)}{(x+x_0)^2 + (y - y_0)^2}\cr
    &-\frac{\lambda}{4\pi\epsilon_0} \frac{2(y+y_0)}{(x-x_0)^2 + (y + y_0)^2} + \frac{\lambda}{4\pi\epsilon_0} \frac{2(y+y_0)}{(x+x_0)^2 + (y + y_0)^2}
\end{align}
It is easily seen that at $x = 0$,
\begin{equation}
    E_y (x=0,y) = 0~.
\end{equation}
Similarly, at $y = 0$, the tangential electric field should be along $x$ axis.
\begin{align} \label{eqn:partial-x}
    E_x(x,y) =&~ - \partial_x \Phi(x,y) \cr
    =&~ \frac{\lambda}{4\pi\epsilon_0} \frac{2(x-x_0)}{(x-x_0)^2 + (y - y_0)^2} - \frac{\lambda}{4\pi\epsilon_0} \frac{2(x+x_0)}{(x+x_0)^2 + (y - y_0)^2}\cr
    &-\frac{\lambda}{4\pi\epsilon_0} \frac{2(x-x_0)}{(x-x_0)^2 + (y + y_0)^2} + \frac{\lambda}{4\pi\epsilon_0} \frac{2(x+x_0)}{(x+x_0)^2 + (y + y_0)^2}
\end{align}
It is easily seen that at $y = 0$,
\begin{equation}
    E_x(x,y=0) = 0~.
\end{equation}

\newpage
\noindent {\bf 2.3} (c) At $y = 0$, the surface charge density is
\begin{equation}
    \sigma(x) = - \epsilon_0 \frac{\partial \Phi}{\partial y}\Bigg|_{y = 0}
\end{equation}
Now using result in \eqref{eqn:partial-y}, we have
\begin{equation}
    \sigma(x) = -\frac{\lambda}{\pi}\left( \frac{y_0}{(x-x_0)^2 + y_0^2} - \frac{y_0}{(x+x_0)^2 + y_0^2} \right)~.
\end{equation}
Then the total charge, per unit length in $z$ on the plane $y = 0, x \ge 0$ is
\begin{align}
    Q_x &= \int_0^\infty dx\, \sigma(x) = -\frac{\lambda}{\pi}\int_0^\infty dx\left( \frac{y_0}{(x-x_0)^2 + y_0^2} - \frac{y_0}{(x+x_0)^2 + y_0^2} \right)\cr
    &= -\frac{\lambda}{\pi} \left(\tan^{-1}\left(\frac{x-x_0}{y_0}\right)\Bigg|_{0}^{\infty} - \tan^{-1}\left(\frac{x+x_0}{y_0}\right)\Bigg|_{0}^{\infty} \right)\cr
    &= - \frac{2}{\pi} \lambda \tan^{-1} \left(\frac{x_0}{y_0}\right)~.
\end{align}

\newpage
\noindent{\bf 2.3} (d) At position far from the origin $\rho \gg \rho_0$, where
\begin{equation}
    \rho = \sqrt{x^2 + y^2}, \quad \rho_0 = \sqrt{x_0^2 + y_0^2}
\end{equation}
We can expand the potential
\begin{align}
    \Phi(x,y) = &~ -\frac{\lambda}{4 \pi \epsilon_0} \ln\left({(x-x_0)^2 + (y - y_0)^2}\right)
    + \frac{\lambda}{4 \pi \epsilon_0} \ln\left({(x+x_0)^2 + (y - y_0)^2}\right)\cr
    &+ \frac{\lambda}{4 \pi \epsilon_0} \ln\left({(x-x_0)^2 + (y + y_0)^2}\right) - \frac{\lambda}{4 \pi \epsilon_0} \ln\left({(x+x_0)^2 + (y + y_0)^2}\right)~.\cr
    =&~ -\frac{\lambda}{4 \pi \epsilon_0} \ln\left(1 + \frac{\rho_0^2 - 2x_0 x - 2y_0 y}{\rho^2}\right) + \frac{\lambda}{4\pi \epsilon_0} \ln\left(1 + \frac{\rho_0^2 + 2x_0x - 2y_0y}{\rho^2}\right)\cr
    &+ \frac{\lambda}{4\pi \epsilon_0} \ln\left(1 + \frac{\rho_0^2 - 2x_0x + 2y_0 y}{\rho^2} \right) - \frac{\lambda}{4\pi \epsilon_0}\ln\left( 1 + \frac{\rho_0^2 + 2x_0x + 2y_0y}{\rho^2}\right)~.\nonumber
\end{align}
Now using $\ln(1 + x) \approx x - \frac{x^2}{2}$ when $x$ is very small, we have
\begin{align}
    \Phi(x,y) \approx -\frac{\lambda}{4\pi \epsilon_0} \Bigg(&\frac{\rho_0^2 - 2x_0 x - 2y_0y}{\rho^2} -\frac{\rho_0^2 + 2x_0x -2y_0y}{\rho^2}  \cr
    &-\frac{\rho_0^2 - 2x_0 x + 2y_0 y}{\rho^2} + \frac{\rho_0^2 + 2x_0 x + 2y_0 y}{\rho^2}\Bigg)\cr
    + \frac{\lambda}{8\pi \epsilon_0} \Bigg[& \left(\frac{\rho_0^2 - 2x_0 x - 2y_0y}{\rho^2}\right)^2 - \left(\frac{\rho_0^2 + 2x_0 x - 2y_0y}{\rho^2}\right)^2 \cr
    &- \left(\frac{\rho_0^2 - 2x_0 x + 2y_0 y}{\rho^2} \right)^2 + \left(\frac{\rho_0^2 + 2x_0 x + 2y_0y}{\rho^2} \right)^2 \Bigg]\cr
    =\frac{\lambda}{8 \pi \epsilon_0 \rho^4} &\times 32 x_0 y_0 x y = \frac{4\lambda}{\pi \epsilon_0} \frac{(x_0 y_0)(xy)}{\rho^4}~.
\end{align}

\newpage
\noindent{\bf 2.7} (a) The general Green function is of the form
\begin{equation}
    G(\vec x, \vec x') = \frac{1}{|\vec x - \vec x'|} + F(\vec x, \vec x')~,
\end{equation}
where $\nabla^2 F(\vec x, \vec x') = 0$. And the Dirichlet boundary condition implies that the Green function has to vanish on $z = 0$ surface. We can view the problem as a unit charge at $x'$, with a conducting plane at $z = 0$. Suppose $\vec x' = (x', y', z')$, we can then put the image charge at $(x',y',-z')$ with negative unit charge. The potential is then given by
\begin{equation}
    \Phi
\end{equation}

\end{document}