\documentclass[12pt]{article}

\usepackage{amsmath,amssymb}
\usepackage{hyperref}

\newcommand{\Z}{\mathbb{Z}}

\begin{document}

\begin{center}
{\bf Phys 5405}\\
HW 2
\end{center}
\textbf{1.14} The Green's theorem is,
\begin{equation}
    \int_V (\phi \nabla^2 \psi - \psi \nabla^2 \phi) d^3 x = \oint_S \left(\phi \frac{\partial \psi}{\partial n} - \psi \frac{\partial \phi}{\partial n} \right) da~.
\end{equation}
Apply this with integration variable $\vec y$ and $\phi = G(\vec x, \vec y)$, $\psi = G(\vec x', \vec y)$ with $\nabla^2_y G(\vec z, \vec y) = -4\pi \delta(\vec y - \vec z)$. Suppose $\vec x$ and $\vec x'$ are inside the volume $V$,then we have
\begin{align}
    {\rm L.H.S.} &= \int_V \Big(G(\vec x, \vec y) \nabla^2_y G(\vec x', \vec y) - G(\vec x', \vec y) \nabla^2_y G(\vec x, \vec y)\Big) d^3 y \cr
    &= -4\pi \int_V \Big(G(\vec x, \vec y) \delta(\vec y - \vec x') - G(\vec x', \vec y) \delta(\vec y - \vec x)\Big) d^3 y \cr
    &= -4\pi [G(\vec x, \vec x') - G(\vec x', \vec x)]~.
\end{align}
Therefore,
\begin{equation}
    G(\vec x, \vec x') - G(\vec x', \vec x) = \frac{1}{4\pi} \oint_S \left(G(\vec x', \vec y) \frac{\partial G(\vec x, \vec y)}{\partial n} - G(\vec x, \vec y) \frac{\partial G(\vec x', \vec y)}{\partial n}\right)da_y~.
\end{equation}

\newpage
\noindent(a) For Dirichlet boundary conditions, we have
\begin{equation}
    G_D(\vec x, \vec y) = 0~, \quad {\rm for}~\vec y~{\rm on}~S~.
\end{equation}
The surface integral vanishes and we have
\begin{equation}
    G_D(\vec x, \vec x') - G_D(\vec x', \vec x) = 0~.
\end{equation}
Therefore, $G_D(\vec x, \vec x') = G_D(\vec x', \vec x)$ and $G_D(\vec x, \vec x')$ is symmetric in $\vec x$ and $\vec x'$.

\newpage
\noindent(b) For Neumann boundary conditions, we have
\begin{equation}
    \frac{\partial G_N}{\partial n'} (\vec x, \vec y) = - \frac{4\pi}{S},\quad {\rm for}~\vec y~{\rm on}~S~.
\end{equation}
Then,
\begin{equation}
    G_N(\vec x, \vec x') - G_N(\vec x', \vec x) = \frac{1}{S} \oint_S \Big( G_N(\vec x, \vec y)  - G_N(\vec x', \vec y)  \Big)da_y~.
\end{equation}
and
\begin{equation}
    G_N(\vec x, \vec x') - \frac{1}{S} \oint_S G_N(\vec x, \vec y) da_y = G_N(\vec x', \vec x) - \frac{1}{S} \oint_S  G_N(\vec x', \vec y) da_y~.
\end{equation}
Therefore, $G_N(\vec x, \vec x')$ is not symmetric in general, but $G_N(\vec x, \vec x') - F(\vec x)$ is symmetric in $\vec x$ and $\vec x'$, where
\begin{equation}
    F(\vec x) = \frac{1}{S} \oint_S G_N(\vec x, \vec y) da_y~.
\end{equation}

\newpage
\noindent(c)
The Neumann boundary solution is
\begin{align}
    \Phi(\vec x) = &~ \frac{1}{4\pi \epsilon_0} \int_V \rho(\vec x') G(\vec x, \vec x') d^3x' + \frac{1}{4\pi} \int_S G(\vec x, \vec x') \frac{\partial \Phi}{\partial n'} da' \cr
    &+\frac{1}{S} \int_{S} \Phi(\vec x') d^3 x'~.
\end{align}
For a transformation, $G_N(\vec x, \vec x') \to G_N(\vec x, \vec x') + F(\vec x)$,
\begin{align}
    \Phi'(\vec x) - \Phi(\vec x) &= \frac{1}{4\pi \epsilon_0}  F(\vec x)\int_V \rho(\vec x') d^3 x' + \frac{1}{4\pi} F(\vec x) \int_S \frac{\partial \Phi}{\partial n'} da' \cr
    &= \frac{F(\vec x)}{4\pi} \left( \int_V \frac{\rho(\vec x')}{\epsilon_0}  d^3 x' + \int_S \nabla \Phi \cdot \hat{n} da;\right)
\end{align}
Since $\vec E = - \nabla \Phi$ and $\int_S \vec E \cdot \hat n da = \int_V \frac{\rho(\vec x')}{\epsilon_0} d^3 x'$, we have
\begin{equation}
    \Phi'(\vec x) = \Phi(\vec x)~.
\end{equation}
Therefore, the addition of $F(\vec x)$ to the Green function does not affect the potential $\Phi(\vec x)$.

\newpage
\noindent{\bf 2.2} (a) Suppose the original charge is at $\vec y$. Substitute the conductor with an image charge at $\vec y'$ with charge $q'$. The potential can be written as
\begin{equation}
\Phi(\vec x) = \frac{1}{4\pi \epsilon_0} \left(\frac{q}{|\vec x - \vec y|} + \frac{q'}{|\vec x - \vec y'|} \right)~.
\end{equation}
The potential has to vanish on conductor, so $\Phi(|\vec x| = a) = 0$
\begin{equation}
    \Phi(|\vec x| = a) = \frac{1}{4\pi \epsilon_0} \left( \frac{q}{\sqrt{a^2 + y^2 - 2ay\cos\theta}} + \frac{q'}{ \sqrt{a^2 + y'^2 -2 ay' \cos \theta}}\right)~.
\end{equation}
Therefore, $q$ and $q'$ should have different signs and
\begin{equation}
    \frac{q^2}{q'^2} = \frac{a^2 + y^2 - 2ay\cos\theta}{a^2 + y'^2 -2 ay' \cos\theta}~,
\end{equation}
which leads to
\begin{equation}
    y' = \frac{a^2}{y}, \quad q' = - \frac{a}{y}q
\end{equation}
By symmetry, $\vec y'$ has to be parallel with $\vec y$, therefore, for $|\vec x| < a$,
\begin{equation}
    \Phi(\vec x') = \frac{q}{4\pi \epsilon_0} \left( \frac{1}{|\vec x - \vec y|} - \frac{a/y}{|\vec x - (a^2/y^2) \vec y|} \right)~.
\end{equation}

\newpage
\noindent(b) The surface charge density can be evaluated as
\begin{align}
    \sigma &= - \epsilon_0 \frac{\partial \Phi}{\partial n} \Bigg|_{|\vec x| = a}  = \epsilon_0 \frac{\partial \Phi}{\partial x} \Bigg|_{|\vec x| = a}\cr
    &= \frac{q}{4\pi} \frac{\partial}{\partial x} \left( \frac{1}{\sqrt{x^2 + y^2 - 2xy\cos \theta}} - \frac{a/y}{\sqrt{x^2 + a^4/y^2 - 2 x a^2 \cos \theta/y} } \right)\Bigg|_{|\vec x| = a}\cr
    &= \frac{q}{4\pi}\left( \frac{-a + y \cos\theta}{(a^2 + y^2 - 2ay\cos \theta)^{3/2}} - \frac{y\cos \theta -y^2/a}{(a^2 + y^2 - 2ay\cos \theta)^{3/2}} \right) \cr
    &= \frac{q}{4\pi} \frac{y^2/a-a}{(a^2 + y^2 - 2ay\cos \theta)^{3/2}} = \frac{q}{4\pi ay} \frac{1-a^2/y^2}{(1+ a^2/y^2 - 2(a/y)\cos \theta)^{3/2}}~.
\end{align}


\newpage
\noindent(c) Using the simple method, we can consider the force acting on $q$, arising from the image charge $q'$. The force is then
\begin{equation}
    \vec F = \frac{1}{4\pi \epsilon_0} q q' \frac{\vec y - \vec y'}{|\vec y - \vec y'|^3} = \frac{q^2 a}{4\pi \epsilon_0 y^3} \frac{\hat y}{\left(1- \frac{a^2}{y^2}\right)^2}~.
\end{equation}

\end{document}