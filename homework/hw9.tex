\documentclass[12pt]{article}
\usepackage{fullpage}

\usepackage{amsmath,amssymb}
\usepackage{hyperref}
\usepackage{empheq}

\newcommand{\Z}{\mathbb{Z}}
\newcommand{\x}{{\bf x}}
\newcommand{\p}{{\bf p}}
\newcommand{\lb}{\left(}
\newcommand{\rb}{\right)}
\newcommand{\mb}{{m \pi/\beta}}

\begin{document}

\begin{center}
{\bf Phys 5405}\\
HW 9 \\
4.2 4.7(a,b) 4.8(a)
\end{center}
\noindent{\bf 4.2} A point dipole with dipole moment $\bf p$ is located at the point ${\bf x}_0$. From the properties of the derivative of a Dirac delta function, show that for calculation of the potential $\Phi$ or the energy of a dipole in an external field, the dipole can be described by an effective charge density
$$
\rho_{\rm eff}(\x) = - \p {\bf \cdot \nabla} \delta(\x - \x_0)
$$

\newpage
\noindent{\bf 4.2}
The potential from the dipole is given by
\begin{equation}
    \Phi(\vec x) = \frac{1}{4\pi \epsilon_0} \int d^3 x' \vec p(\vec x') \cdot \nabla'\left(\frac{1}{|\vec x - \vec x'|}\right)~.
\end{equation}
The distribution of the dipole is a delta function,
\begin{equation}
    \vec p(\vec x') = \vec p \, \delta(\vec x' - \vec x_0)~.
\end{equation}
Plug it into the formula of potential and integrate by parts,
\begin{equation}
    \Phi(\vec x) = -\frac{1}{4\pi \epsilon_0} \int d^3 x' \vec p \,\cdot \nabla' \delta(\vec x' - \vec x_0) \frac{1}{|\vec x - \vec x'|} \equiv \frac{1}{4\pi \epsilon_0} \int d^3x' \frac{\rho_{\rm eff}(\vec x')}{|\vec x - \vec x'|}~.
\end{equation}
Therefore, the dipole can be described by an effective charge density,
\begin{equation}
    \boxed{
    \rho_{\rm eff}(\vec x) = - \vec p\, \cdot \nabla \delta(\vec x - \vec x_0)~.
    }
\end{equation}

\newpage
\noindent{\bf 4.7} A localized distribution of charge has a charge density
$$
\rho({\bf r}) = \frac{1}{64 \pi} r^2 e^{-r} \sin^2\theta~
$$

\noindent{\bf (a)} Make a multipole expansion of the potential due to this charge density and determine all the nonvanishing multipole moments. Write down the potential at large distances as a finite expansion in Legendre polynomials.

\noindent{\bf (b)} Determine the potential explicitly at any point in space, and show that near the origin, correct to $r^2$ inclusive,
$$
\Phi({\bf r}) \simeq \frac{1}{4\pi \epsilon_0} \left[\frac{1}{4} - \frac{r^2}{120} P_2(\cos\theta)\right]
$$

\newpage
\noindent{\bf 4.7 (a)} The multipole expansion is
\begin{equation}
    \Phi(\vec x) = \frac{1}{4\pi \epsilon_0} \sum_{l = 0}^\infty \sum_{m = -l}^l \frac{4\pi}{2l+1} q_{lm} \frac{Y_{lm}(\theta, \phi)}{r^{l+1}}~,
\end{equation}
with multipole moments,
\begin{align}
    q_{lm} &= \int Y^*_{lm}(\theta, \phi) r^l \rho(\vec x) d^3 x\cr
    &=\int Y_{lm}^*(\theta, \phi) r^l \left(\frac{1}{64\pi} r^2 e^{-r} \sin^2 \theta\right) r^2 \sin\theta dr d\theta d\phi
\end{align}
The spherical harmonics are related to the associated Legendre polynomials as
\begin{equation}
    Y_{lm}(\theta, \phi) = \sqrt{\frac{2l+1}{4\pi} \frac{(l-m)!}{(l+m)!}} P_l^m(\cos \theta) e^{im\phi}~.
\end{equation}
Consider the integral $\int_0^{2\pi} e^{im\phi} d\phi$, it vanishes for $m \neq 0$. Therefore,
\begin{align}
    q_{lm} &= \delta_{m,0} \int \sqrt{\frac{2l+1}{4\pi}} P_l(\cos\theta)r^l \left(\frac{1}{64\pi} r^2 e^{-r} \sin^2 \theta\right) r^2 \sin\theta dr d\theta d\phi \cr
    &= \frac{2\pi}{64\pi}\sqrt{\frac{2l+1}{4\pi}} \delta_{m,0} \left(\int dr\, r^{l+4} e^{-r} \right) \left(\int d\theta\, P_l(\cos \theta) \sin^3\theta\right)
\end{align}
By the definition of Gamma functions,
\begin{equation}
    \boxed{
    \int_0^\infty dr\, r^{l+4} e^{-r} = \Gamma(l+5) = (l+4)!
    }
\end{equation}
For another integral, we change the variable as $x = \cos \theta$, and we also make use of the identity, $x^2 = \frac{2}{3} P_2(x) + \frac{1}{3}$, then the integral is now given by
\begin{equation}\label{eqn:a-integral}
    \boxed{
    \int_{-1}^1 dx\, P_l(x)(1-x^2) = \frac{2}{3}\int_{-1}^1 dx\, P_l(x) (1 - P_2(x)) = \frac{2}{3} \left(2 \delta_{l,0} - \frac{2}{5} \delta_{l,2}\right)~.
    }
\end{equation}
Therefore, the only non-zero multipole moments are
\begin{equation}
    \boxed{
    q_{00} = \frac{1}{2} \frac{1}{\sqrt{\pi}}~, \quad  q_{20} = -3\sqrt{\frac{5}{\pi}}~.
    }
\end{equation}
Therefore, the potential is given by
\begin{equation}
    \boxed{
    \Phi = \frac{1}{4\pi \epsilon_0} \left[\frac{1}{r} - \frac{6 P_2(\cos \theta)}{r^3}\right] = \frac{1}{4\pi \epsilon_0} \left[\frac{P_0(\cos\theta)}{r} - \frac{6 P_2(\cos \theta)}{r^3}\right]~.
    }
\end{equation}

\newpage
The potential is calculated as
\begin{equation}
    \Phi = \frac{1}{4\pi \epsilon_0} \int \frac{\rho(\vec x')}{|\vec x - \vec x'|} d^3 x' = \frac{1}{4\pi \epsilon_0} \frac{1}{64 \pi} \int \frac{r'^2 e^{-r'}\sin^2\theta' }{|\vec x - \vec x'|} r'^2 \sin\theta' dr' d\theta' d\phi'~.
\end{equation}
Now we use Jackson (3.70),
\begin{equation}
    \frac{1}{|\vec x - \vec x'|} = 4\pi \sum_{l = 0}^\infty \sum_{m = -l}^l \frac{1}{2l+1} \frac{r^l_<}{r^{l+1}_>} Y_{lm}^*(\theta', \phi') Y_{lm}(\theta, \phi)~.
\end{equation}
Due to azimuthal symmetry of the charge distribution, we only need to consider $m = 0$,
\begin{equation}
    \boxed{
    \frac{1}{|\vec x - \vec x'|} = \sum_{l = 0}^\infty \frac{r^l_<}{r_>^{l+1}} P_l(\cos \theta) P_l(\cos \theta')~.
    }
\end{equation}
Therefore, the potential is
\begin{align}
    \Phi &= \sum_{l = 0}^\infty \frac{1}{4\pi \epsilon_0} \frac{1}{64\pi} \int r'^2 e^{-r'} \sin^2 \theta' \frac{r^l_<}{r_>^{l+1}} P_l(\cos \theta) P_l(\cos \theta')r'^2 \sin\theta' dr' d\theta' d\phi' \cr
    &= \sum_{l = 0}^\infty \frac{1}{4\pi \epsilon_0} \frac{1}{32}  P_l(\cos \theta) \left(\int r'^4 e^{-r'}  \frac{r^l_<}{r_>^{l+1}}  dr'\right)  \left(\int_{-1}^1 dx\, P_l(x) (1 - x^2)\right)~.
\end{align}
The second integral is already calculated in \eqref{eqn:a-integral}, we copy it here for reference,
\begin{equation}
    \int_{-1}^1 dx\, P_l(x)(1-x^2)  = \frac{2}{3} \left(2 \delta_{l,0} - \frac{2}{5} \delta_{l,2}\right)~.
\end{equation}
Therefore,
\begin{equation}
    \boxed{
    \Phi =  \frac{1}{4\pi \epsilon_0} \frac{1}{24} \left(\int r'^4 e^{-r'}  \frac{1}{r_>}  dr'\right) - \frac{1}{4\pi \epsilon_0} \frac{1}{120}\left(\int r'^4 e^{-r'}  \frac{r_<^2}{r_>^3} dr'\right) P_2(\cos \theta)~.
    }
\end{equation}
First,
\begin{align}
    \int_0^\infty  dr'\, r'^4 e^{-r'}  \frac{1}{r_>} &= \int_0^r dr'\, r'^4 e^{-r'} \frac{1}{r} + \int_{r}^{\infty} dr' \, r'^4 e^{-r'} \frac{1}{r'} \cr
    &= \boxed{\frac{1}{r} e^{-r} (-24 + 24 e^r - 18 r - 6 r^2 - r^3)}~.
\end{align}
Second,
\begin{align}
    \int_0^\infty dr'\, r'^4 e^{-r'}  \frac{r_<^2}{r_>^3} &= \int_0^r dr' \, r'^4 e^{-r'}  \frac{r'^2}{r^3} + \int_r^\infty dr' \, r'^4 e^{-r'}  \frac{r^2}{r'^3} \cr
    &= \boxed{5 \frac{1}{r^3}e^{-r}(-144 + 144 e^r - 144 r - 72 r^2 - 24 r^3 - 6r^4 - r^5)}~.
\end{align}
Therefore, the exact potential is given by
\begin{empheq}[box=\fbox]{align}
    \Phi =&~ \frac{1}{4\pi \epsilon_0 } \frac{e^{-r}}{r}\left( e^r - 1 - \frac 34 r - \frac 14 r^3 - \frac{1}{24} r^3\right) \cr
    &- \frac{1}{4\pi \epsilon_0}\frac{e^{-r}}{r^3}\left(6 e^r - 6 - 6r - 3r^2 - r^2 - \frac{1}{4}r^4 - \frac{1}{24} r^5\right) P_2(\cos \theta)~.
\end{empheq}
Near the origin, we expand,
\begin{equation}
    e^{-r} = 1 - r + \frac{1}{2}r^2 - \frac{1}{6}r^3 + \frac{1}{24}r^4 - \frac{1}{120}r^5 + \cdots~.
\end{equation}
Then, up to $r^2$ order,
\begin{equation}
    \Phi \approx \frac{1}{4\pi \epsilon_0} \frac{1}{4} - \frac{1}{4\pi \epsilon_0} \frac{1}{120} r^2 P_2(\cos \theta) = \boxed{\frac{1}{4\pi \epsilon_0} \left[\frac{1}{4} - \frac{r^2}{120} P_2(\cos \theta)\right]}~.
\end{equation}


\newpage
\noindent{\bf 4.8} A very long, right circular, cylindrical shell of dielectric constant $\epsilon/\epsilon_0$ and inner and outer radii $a$ and $b$, respectively, is placed in a previously uniform electric field $E_0$ with its axis perpendicular to the field. The medium inside and outside the cylinder has a dielectric constant of unity.

\noindent{\bf (a)} Determine the potential and electric field in the three regions, neglecting end effects.

\newpage
\noindent{\bf 4.8 (a)} Neglecting end effects, we can think that the cylinder is infinitely long. Since the system is translation invariant along $z$-axis, we can treat it as a two dimensional problem. Denote the distance from the axis by $\rho$ and the angle by $\phi$. The general solution is
\begin{equation}
    \Phi(\rho, \phi) = a_0 + b_0 \ln\rho + \sum_{n = 1}^\infty a_n \rho^n \sin(n \phi + \alpha_n) + \sum_{n = 1}^\infty b_n \rho^{-n} \sin(n \phi + \beta_n)~.
\end{equation}
Since this system satisfies $\Phi(\rho, \phi) = \Phi(\rho, -\phi)$, we can write the general solution as
\begin{equation}
    \Phi(\rho, \phi) = a_0 + b_0 \ln\rho + \sum_{n =1}^\infty (a_n \rho^n + b_n \rho^{-n}) \cos(n \phi) ~.
\end{equation}
Now consider finiteness. For $\rho < a$,
\begin{equation}
    \Phi_1(\rho, \phi) = a_0 + \sum_{n = 1}^\infty a_n \rho^n \cos(n \phi)~.
\end{equation}
For $a < \rho < b$,
\begin{equation}
    \Phi_2(\rho, \phi) = b_0 + c_0 \ln\rho + \sum_{n=1}^\infty (b_n \rho^n + c_n \rho^{-n}) \cos(n \phi)~.
\end{equation}
For $\rho > b$,
\begin{equation}
    \Phi_3(\rho, \phi) = d_0 + \sum_{n=1}^\infty d_n \rho^{-n} \cos(n \phi) - E_0 \rho \cos \phi~,
\end{equation}
where the last term is added due to the constant electric field at infinity.

The boundary conditions are the tangential $E$ is continuous,
\begin{equation}\label{eqn:tangential-E}
    -\frac{1}{a} \frac{\partial \Phi_1}{\partial \phi} \Bigg|_{\rho = a} = - \frac{1}{a} \frac{\partial \Phi_2}{\partial \phi} \Bigg|_{\rho = a}~, \quad - \frac{1}{b}\frac{\partial \Phi_2}{\partial \phi} \Bigg|_{\rho = b} = - \frac{1}{b} \frac{\partial \Phi_3}{\partial \phi} \Bigg|_{\rho =b}~.
\end{equation}
Also, the normal $D$ is continuous,
\begin{equation}\label{eqn:normal-D}
    -\epsilon_0 \frac{\partial \Phi_1}{\partial \rho} \Bigg|_{\rho = a} = - \epsilon \frac{\partial \Phi_2}{\partial \rho} \Bigg|_{\rho = a}~, \quad - \epsilon \frac{\partial \Phi_2}{\partial \rho} \Bigg|_{\rho = b} =  - \epsilon_0 \frac{\partial \Phi_3}{\partial \rho} \Bigg|_{\rho =b}~.
\end{equation}
Now we calculate the derivatives of $\Phi_1$,
\begin{equation}
    \frac{\partial \Phi_1}{\partial \phi} = -\sum_{n = 1}^\infty n a_n \rho^n \sin(n \phi)~, \quad \frac{\partial \Phi_1}{\partial \rho} = \sum_{n=1}^\infty n a_n \rho^{n-1} \cos(n \phi)~.
\end{equation}
Also, the derivatives of $\Phi_2$,
\begin{equation}
    \frac{\partial \Phi_2}{\partial \phi} = - \sum_{n=1}^\infty n(b_n \rho^n + c_n \rho^{-n}) \sin(n \phi)~, \quad \frac{\partial \Phi_2}{\partial \rho} = \frac{c_0}{\rho} + \sum_{n=1}^\infty (n b_n \rho^{n-1} - n c_n \rho^{-n-1}) \cos(n \phi)~.
\end{equation}
And the derivatives of $\Phi_3$,
\begin{equation}
    \frac{\partial \Phi_3}{\partial \phi} = -\sum_{n = 1}^\infty n d_n \rho^{-n} \sin(n \phi) + E_0 \rho \sin\phi~, \quad \frac{\partial \Phi_3}{\partial \rho} = -\sum_{n=1}^\infty n d_n \rho^{-n-1} \cos(n \phi) - E_0 \cos \phi~.
\end{equation}

From \eqref{eqn:tangential-E}, we can derive that,
\begin{empheq}[box=\fbox]{align}
    c_n &= (a_n - b_n) a^{2n}~, \quad n\ge 1~.\\
    b_n &= (d_n - c_n) b^{-2n} ~, \quad n\ge 2~.\\
    b_1 &= (d_1 - c_1) b^{-2} - E_0~.
\end{empheq}

From \eqref{eqn:normal-D}, we can derive that,
\begin{empheq}[box=\fbox]{align}
    c_0 &= 0~.\\
    a_n a^{2n} & = \frac{\epsilon}{\epsilon_0} (b_n a^{2n} - c_n)~, \quad n\ge 1~.\\
    -\epsilon(b_n b^{n-1} - c_n b^{-n-1}) &= \epsilon_0 d_n b^{-n-1} ~, \quad n\ge 2~.\\
    - \epsilon (b_1 - c_1 b^{-2}) &= \epsilon_0 d_1 b^{-2} + \epsilon_0 E_0~.
\end{empheq}
For $n \ge 2$, we can derive that
\begin{equation}
    \boxed{
    a_n = b_n = c_n = d_n = 0~.
    }
\end{equation}
For $n = 1$,
We can derive that
\begin{equation}
    a_1 = \frac{2 \epsilon}{\epsilon + \epsilon_0} b_1~, c_1 = \frac{\epsilon - \epsilon_0}{\epsilon + \epsilon_0} a^2 b_1~, d_1 = \frac{2\epsilon}{\epsilon + \epsilon_0} \frac{\epsilon - \epsilon_0}{\epsilon + \epsilon_0} a^2 b_1 + \frac{\epsilon - \epsilon_0}{\epsilon + \epsilon_0} E_0 b^2~,
\end{equation}
with $b_1$ given by
\begin{equation}
    b_1 = \left[\left(\frac{\epsilon - \epsilon_0}{\epsilon + \epsilon_0}\right)^2 \frac{a^2}{b^2} - 1\right]^{-1} \frac{2\epsilon_0}{\epsilon + \epsilon_0} E_0~.
\end{equation}
Therefore, we obtain
\begin{empheq}[box=\fbox]{align}
    a_1 &= \frac{4 \epsilon \epsilon_0 b^2 E_0}{a^2(\epsilon- \epsilon_0)^2 - b^2(\epsilon + \epsilon_0)^2}~.\\
    b_1 &= \frac{2\epsilon_0 (\epsilon + \epsilon_0) b^2 E_0}{a^2(\epsilon- \epsilon_0)^2 - b^2(\epsilon + \epsilon_0)^2}~.\\
    c_1 &= \frac{2\epsilon_0(\epsilon-\epsilon_0) a^2 b^2 E_0}{a^2(\epsilon- \epsilon_0)^2 - b^2(\epsilon + \epsilon_0)^2}~.\\
    d_1 &= \frac{(\epsilon^2 - \epsilon_0^2)(a^2 - b^2)b^2 E_0 }{a^2(\epsilon- \epsilon_0)^2 - b^2(\epsilon + \epsilon_0)^2}~.
\end{empheq}
Now the continuity of the potential leads
\begin{equation}
    a_0 = b_0 = d_0~.
\end{equation}
We can drop this constant in the potential. and now we have
\begin{equation}
    \Phi(\rho, \phi) =
    \begin{cases}
        \frac{4 \epsilon \epsilon_0 b^2 E_0}{a^2(\epsilon- \epsilon_0)^2 - b^2(\epsilon + \epsilon_0)^2} \rho \cos \phi~, & \rho < a\\\\
        \frac{2\epsilon_0 (\epsilon + \epsilon_0) b^2 E_0}{a^2(\epsilon- \epsilon_0)^2 - b^2(\epsilon + \epsilon_0)^2} \rho \cos \phi + \frac{2\epsilon_0(\epsilon-\epsilon_0) a^2 b^2 E_0}{a^2(\epsilon- \epsilon_0)^2 - b^2(\epsilon + \epsilon_0)^2} \rho^{-1} \cos \phi~, & a < \rho < b\\\\
        \frac{(\epsilon^2 - \epsilon_0^2)(a^2 - b^2)b^2 E_0 }{a^2(\epsilon- \epsilon_0)^2 - b^2(\epsilon + \epsilon_0)^2} \rho^{-1} \cos\phi -E_0 \rho \cos \phi~, & \rho >b
    \end{cases}
\end{equation}

\end{document}