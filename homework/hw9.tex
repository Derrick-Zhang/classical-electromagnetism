\documentclass[12pt]{article}
\usepackage{fullpage}

\usepackage{amsmath,amssymb}
\usepackage{hyperref}

\newcommand{\Z}{\mathbb{Z}}
\newcommand{\x}{{\bf x}}
\newcommand{\p}{{\bf p}}
\newcommand{\lb}{\left(}
\newcommand{\rb}{\right)}
\newcommand{\mb}{{m \pi/\beta}}

\begin{document}

\begin{center}
{\bf Phys 5405}\\
HW 9 \\
4.2 4.7(a,b) 4.8(a)
\end{center}
\noindent{\bf 4.2} A point dipole with dipole moment $\bf p$ is located at the point ${\bf x}_0$. From the properties of the derivative of a Dirac delta function, show that for calculation of the potential $\Phi$ or the energy of a dipole in an external field, the dipole can be described by an effective charge density
$$
\rho_{\rm eff}(\x) = - \p {\bf \cdot \nabla} \delta(\x - \x_0)
$$

\newpage
\noindent{\bf 4.2}
The potential from the dipole is given by
\begin{equation}
    \Phi(\vec x) = \frac{1}{4\pi \epsilon_0} \int d^3 x' \vec p(\vec x') \cdot \nabla'\left(\frac{1}{|\vec x - \vec x'|}\right)~.
\end{equation}
The distribution of the dipole is a delta function,
\begin{equation}
    \vec p(\vec x') = \vec p \, \delta(\vec x' - \vec x_0)~.
\end{equation}
Plug it into the formula of potential and integrate by parts,
\begin{equation}
    \Phi(\vec x) = -\frac{1}{4\pi \epsilon_0} \int d^3 x' \vec p \,\cdot \nabla' \delta(\vec x' - \vec x_0) \frac{1}{|\vec x - \vec x'|} \equiv \frac{1}{4\pi \epsilon_0} \int d^3x' \frac{\rho_{\rm eff}(\vec x')}{|\vec x - \vec x'|}~.
\end{equation}
Therefore, the dipole can be described by an effective charge density,
\begin{equation}
    \boxed{
    \rho_{\rm eff}(\vec x) = - \vec p\, \cdot \nabla \delta(\vec x - \vec x_0)~.
    }
\end{equation}

\newpage
\noindent{\bf 4.7} A localized distribution of charge has a charge density
$$
\rho({\bf r}) = \frac{1}{64 \pi} r^2 e^{-r} \sin^2\theta~
$$

\noindent{\bf (a)} Make a multipole expansion of the potential due to this charge density and determine all the nonvanishing multipole moments. Write down the potential at large distances as a finite expansion in Legendre polynomials.

\noindent{\bf (b)} Determine the potential explicitly at any point in space, and show that near the origin, correct to $r^2$ inclusive,
$$
\Phi({\bf r}) \simeq \frac{1}{4\pi \epsilon_0} \left[\frac{1}{4} - \frac{r^2}{120} P_2(\cos\theta)\right]
$$

\newpage
\noindent{\bf 4.7 (a)} The multipole expansion is
\begin{equation}
    \Phi(\vec x) = \frac{1}{4\pi \epsilon_0} \sum_{l = 0}^\infty \sum_{m = -l}^l \frac{4\pi}{2l+1} q_{lm} \frac{Y_{lm}(\theta, \phi)}{r^{l+1}}~,
\end{equation}
with multipole moments,
\begin{align}
    q_{lm} &= \int Y^*_{lm}(\theta, \phi) r^l \rho(\vec x) d^3 x\cr
    &=\int Y_{lm}^*(\theta, \phi) r^l \left(\frac{1}{64\pi} r^2 e^{-r} \sin^2 \theta\right) r^2 \sin\theta dr d\theta d\phi
\end{align}
The spherical harmonics are related to the associated Legendre polynomials as
\begin{equation}
    Y_{lm}(\theta, \phi) = \sqrt{\frac{2l+1}{4\pi} \frac{(l-m)!}{(l+m)!}} P_l^m(\cos \theta) e^{im\phi}~.
\end{equation}
Consider the integral $\int_0^{2\pi} e^{im\phi} d\phi$, it vanishes for $m \neq 0$. Therefore,
\begin{align}
    q_{lm} &= \delta_{m,0} \int \sqrt{\frac{2l+1}{4\pi}} P_l(\cos\theta)r^l \left(\frac{1}{64\pi} r^2 e^{-r} \sin^2 \theta\right) r^2 \sin\theta dr d\theta d\phi \cr
    &= \frac{2\pi}{64\pi}\sqrt{\frac{2l+1}{4\pi}} \delta_{m,0} \left(\int dr\, r^{l+4} e^{-r} \right) \left(\int d\theta\, P_l(\cos \theta) \sin^3\theta\right)
\end{align}
By the definition of Gamma functions,
\begin{equation}
    \boxed{
    \int_0^\infty dr\, r^{l+4} e^{-r} = \Gamma(l+5) = (l+4)!
    }
\end{equation}
For another integral, we change the variable as $x = \cos \theta$, and we also make use of the identity, $x^2 = \frac{2}{3} P_2(x) + \frac{1}{3}$, then the integral is now given by
\begin{equation}
    \boxed{
    \int_{-1}^1 dx\, P_l(x)(1-x^2) = \frac{2}{3}\int_{-1}^1 dx\, P_l(x) (1 - P_2(x)) = \frac{2}{3} \left(2 \delta_{l,0} - \frac{2}{5} \delta_{l,2}\right)~.
    }
\end{equation}
Therefore, the only non-zero multipole moments are
\begin{equation}
    \boxed{
    q_{00} = \frac{1}{2} \frac{1}{\sqrt{\pi}}~, \quad  q_{20} = -3\sqrt{\frac{5}{\pi}}~.
    }
\end{equation}
Therefore, the potential is given by
\begin{equation}
    \boxed{
    \Phi = \frac{1}{4\pi \epsilon_0} \left[\frac{1}{r} - \frac{6 P_2(\cos \theta)}{r^3}\right] = \frac{1}{4\pi \epsilon_0} \left[\frac{P_0(\cos\theta)}{r} - \frac{6 P_2(\cos \theta)}{r^3}\right]~.
    }
\end{equation}

\newpage
\noindent{\bf 4.8} A very long, right circular, cylindrical shell of dielectric constant $\epsilon/\epsilon_0$ and inner and outer radii $a$ and $b$, respectively, is placed in a previously uniform electric field $E_0$ with its axis perpendicular to the field. The medium inside and outside the cylinder has a dielectric constant of unity.

\noindent{\bf (a)} Determine the potential and electric field in the three regions, neglecting end effects.

\end{document}