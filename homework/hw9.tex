\documentclass[12pt]{article}
\usepackage{fullpage}

\usepackage{amsmath,amssymb}
\usepackage{hyperref}

\newcommand{\Z}{\mathbb{Z}}
\newcommand{\x}{{\bf x}}
\newcommand{\p}{{\bf p}}
\newcommand{\lb}{\left(}
\newcommand{\rb}{\right)}
\newcommand{\mb}{{m \pi/\beta}}

\begin{document}

\begin{center}
{\bf Phys 5405}\\
HW 9 \\
4.2 4.7(a,b) 4.8(a)
\end{center}
\noindent{\bf 4.2} A point dipole with dipole moment $\bf p$ is located at the point ${\bf x}_0$. From the properties of the derivative of a Dirac delta function, show that for calculation of the potential $\Phi$ or the energy of a dipole in an external field, the dipole can be described by an effective charge density
$$
\rho_{\rm eff}(\x) = - \p {\bf \cdot \nabla} \delta(\x - \x_0)
$$

\newpage
\noindent{\bf 4.2}
The potential from the dipole is given by
\begin{equation}
    \Phi(\vec x) = \frac{1}{4\pi \epsilon_0} \int d^3 x' \vec p(\vec x') \cdot \nabla'\left(\frac{1}{|\vec x - \vec x'|}\right)~.
\end{equation}
The distribution of the dipole is a delta function,
\begin{equation}
    \vec p(\vec x') = \vec p \, \delta(\vec x' - \vec x_0)~.
\end{equation}
Plug it into the formula of potential and integrate by parts,
\begin{equation}
    \Phi(\vec x) = -\frac{1}{4\pi \epsilon_0} \int d^3 x' \vec p \,\cdot \nabla' \delta(\vec x' - \vec x_0) \frac{1}{|\vec x - \vec x'|} \equiv \frac{1}{4\pi \epsilon_0} \int d^3x' \frac{\rho_{\rm eff}(\vec x')}{|\vec x - \vec x'|}~.
\end{equation}
Therefore, the dipole can be described by an effective charge density,
\begin{equation}
    \rho_{\rm eff}(\vec x) = - \vec p\, \cdot \nabla \delta(\vec x - \vec x_0)~.
\end{equation}

\newpage
\noindent{\bf 4.7} A localized distribution of charge has a charge density
$$
\rho({\bf r}) = \frac{1}{64 \pi} r^2 e^{-r} \sin^2\theta~
$$

\noindent{\bf (a)} Make a multipole expansion of the potential due to this charge density and determine all the nonvanishing multipole moments. Write down the potential at large distances as a finite expansion in Legendre polynomials.

\noindent{\bf (b)} Determine the potential explicitly at any point in space, and show that near the origin, correct to $r^2$ inclusive,
$$
\Phi({\bf r}) \simeq \frac{1}{4\pi \epsilon_0} \left[\frac{1}{4} - \frac{r^2}{120} P_2(\cos\theta)\right]
$$

\newpage
\noindent{\bf 4.8} A very long, right circular, cylindrical shell of dielectric constant $\epsilon/\epsilon_0$ and inner and outer radii $a$ and $b$, respectively, is placed in a previously uniform electric field $E_0$ with its axis perpendicular to the field. The medium inside and outside the cylinder has a dielectric constant of unity.

\noindent{\bf (a)} Determine the potential and electric field in the three regions, neglecting end effects.

\end{document}